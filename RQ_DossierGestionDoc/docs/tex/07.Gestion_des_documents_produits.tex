\section{Gestion des documents produits}
\subsection{Etat des documents et gestion des modifications}
Les documents dits "drafts" sont placés dans un dossier prévu à cet effet dans
un sous-dossier correspondant à la partie concernée. 
Les livrables intermédiaires et finaux sont placés dans les dossiers 
correspondants au livrable.
L'avancement d'un livrable est suivi sur le tableau de bord. Le rédacteur se
doit de mettre à jour son avancement sur le tableau de bord.
Un document n'est plus à l'état d'ébauche lorsqu'il a été validé et vérifié par 
le chef de projet et le responsable qualité.
En plus du tableau de bord, il est nécessaire d'éditer le fichier de suivi
pour indiquer une modification et donc une nouvelle version du document. Il est
conseillé de faire ceci à chaque "commit".

\subsection{Vérification/Validation}
Ces étapes interviennent lorsque toutes les parties d'un livrable sont 
déclarées terminées sur le tableau de bord.
La vérification et la validation sont inclus dans le fichier 0.b.suivi.tex de 
chaque livrable. Si la vérification entraîne des demandes de modification, 
ces demandes seront ajoutées sous forme de remarques sur le tableau de bord.
Les rédacteurs se doivent de vérifier l'état de vérification et de validation
des documents dont ils sont responsables.

\subsection{Gestion des versions}
A chaque fois qu'un rédacteur finit de modifier un document et qu'il décide 
de faire un "commit" il doit modifier le fichier 0.b.suivi.tex en y précisant 
son nom, la date et le numéro de version.

\subsection{Gestion des sauvegardes}
Tous les documents du projet sont contenus sur un dépôt Git. 
La gestion de versionnement et de conflit est généralement automatique. 
Chaque membre du projet se doit de maîtriser les bases de l'outil Git.
La Best Practice sur l'utilisation de git est de faire un "commit" à chaque 
sous-tâche effectuée en précisant celle-ci en commentaire de "commit". Ceci 
pour des soucis de suivi évidents. Cela permet aussi des gestions de conflit
plus faciles.
Il est fortement conseillé de récupérer le travail des autres régulièrement
pour pouvoir gérer les conflits localement.
C'est-à-dire utiliser régulièrement la commande "git pull".
Si il n'y a pas de conflit, vous pouvez utiliser la commande "git push"
pour mettre en ligne vos différents "commit".
Bien que Git l'interdit, il peut arriver que cela fonctionne, c'est pourquoi
il est formellement interdit d'utiliser une commande "push" sans avoir utiliser
la commande "pull" au préalable.
