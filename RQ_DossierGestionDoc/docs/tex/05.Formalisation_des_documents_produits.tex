\section{Formalisation des documents produits}
\subsection{Processus de création d'un nouveau document}
Chaque nouveau document à l'exception des drafts est créé par le responsable 
qualité ou le chef de projet. Il s'agira toujours d'un ensemble de fichiers au 
format Latex.
Les documents sont découpés en plusieurs fichiers correspondants chacun à une
section du document.

\subsection{Présentation des documents}
\subsubsection{Draft}
Les drafts sont placés dans un dossier spécifique et ne sont soumis à aucune 
règle structurelle. Il est par ailleurs conseillé de respecter la règles des 
5 lignes : une idée doit pouvoir être transmise en 5 lignes.
Les documents de type draft doivent contenir l'auteur, la date de rédaction, 
le positionnement dans le projet, la philosophie du draft, l'objectif et les 
idées directrices.
Les drafts ne sont pas contrôlés par le Responsable Qualité, il est donc libre
au rédacteur de choisir comment le rédiger. Cependant, il est conseillé
d'utiliser un format qui sera facilement lisible par le chef de projet.

\subsubsection{Livrables intermédiaires, livrables finaux}
Les livrables intermédiaires et finaux devront respecter les plans type donnés 
en annexe.
Les rédacteurs doivent donc insérer leur texte dans les sections, sous-section
ou sous-sous-section prévues à cet effet. Ils peuvent se référer au plan-type
et/ou au tableau de bord pour l'identifier.
Tous les livrables contiennent des informations de suivi qui doivent êtres 
mises à jour à chaque modification par un rédacteur. La validation et 
vérification sont aussi contenues dans le fichier de suivi 00.suivi.tex.
