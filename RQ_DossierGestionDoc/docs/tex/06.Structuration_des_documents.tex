\section{Structuration des documents}
\subsection{Informations de la page de garde}
On trouve sur la page de garde premi�rement le titre et l'auteur ainsi que le
logo INSA.
La page suivant la page de garde contient les informations de versionnement du 
document. Il s'agit du fichier 0.b.Suivi.tex pr�sent dans chaque dossier de 
livrable. Cette page est g�n�rique et g�n�r�e par le Responsable Qualit� ou le 
Chef de Projet. 

\subsection{Informations de pied de page}

Le pied de page reprend le num�ro de page, le nom du livrable, l'auteur et le 
nom du projet.

\subsection{Informations au coeur du document}

La structuration des paragraphes est libre aux r�dacteurs. La structuration 
globale du document est mise en place par le Responsable Qualit� au format 
Latex. Latex met automatiquement en page et r�alise la num�rotation et le
d�coupage. Ce n'est donc pas aux r�dacteurs de s'en occuper. Ils doivent
seulement veiller � ins�rer le texte au bon endroit dans les fichiers Latex.
