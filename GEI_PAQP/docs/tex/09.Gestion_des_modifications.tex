\section{Gestions des modifications}

\subsection{Evolutions fonctionnelles ou réglementaires}

    \subsubsection{Responsabilités}
    Le Chef de Projet avec le Responsable Qualité sont les seuls en pouvoir 
    d'accepter une modification.

    \subsubsection{Procédure de gestion}
    Lorsqu'une modification peut être indispensable, l'acteur ayant besoin de 
cette modification envoie un AVP.
    Cet AVP est reçu par le Responsable Qualité qui consulte le Chef de Projet.
    
    \par{En cas de refus de la modification}
        Il est alors précisé au demandeur qu'il est possible et qu'il doit 
réaliser son travail sans cette modification.
    \par{En cas d'acceptation}
        Consulter "Corrections ou adaptations" ci-après dans ce même chapitre.

    \subsubsection{Outils de gestion des évolutions}

\subsection{Corrections ou adaptations}
    
    \subsubsection{Responsabilités}

    \subsubsection{Procédure de gestion}
    En cas de correction au d'adaptation à faire, une commission composée du 
RQ, du CdP et des différentes personnes supposément concernées par le projet se 
réunit et analysent l'impact de la correction sur le sous-projet et l'ensemble 
du projet.
    Un plan de mise en place de la correction est alors mis en place.

    \subsubsection{Outils de gestion des corrections ou adaptations}

        
