\section{Introduction}

%Nous devons définir des axes d'améliorations, afin de partir de ce
%qui existe pour obtenir une solution satisfaisant les exigences
%fonctionnelles et non fonctionnelles. 
%Nous avons ainsi défini plusieurs axes principaux, ainsi que des
%axes secondaires, moins importants.

Au vu de l'étude de faisabilité, nous pouvons désormais définir
des axes d'amélioration de l'existant, principaux et secondaires;
définir les principaux choix technologiques, ce qui va nous permettre
de faire une description des exigences fonctionnelles et non fonctionnelles
du nouveau système.

Nous allons également pouvoir réaliser une analyse des impacts vis-à-vis
des besoins auxquels la solution que nous proposons doit répondre
immédiatement, ainsi que l'impact des besoins auxquels on devra répondre à
plus long terme; de manière à faciliter l'évolutivité fonctionnelle du
futur système.

Enfin, on va montrer en quoi le nouveau système répond aux services attendus. 
