\section{Présentation du problème}
%But cause intérêt finalité utilisateurs concernés
%Formulation des besoins généraux exploitation ergonomie expérience
%Portée, Dvpt, Mise en oeuvre , Organisation de la maintenance
%Limites

Ce cahier des charges va permettre de mettre les bases pour établir la mise 
en place de la formation pour les utilisateurs finaux c'est à dire : 

%Acteurs
Ce document vise les formations aux acteurs suivants : 
\begin{itemize}
	\item le COPEVUE commenditaire du projet,
	\item les propriétaires dont les sites seront surveillés,
	\item les sociétés d'intervention qui interviennent sur les sites des 
	propriétaires pour de la maintenance.
\end{itemize}

Les utilisateurs ont des origines diverses aussi bien d'un point de leurs parcours
 académique que de leurs expériences, et leurs habitudes à utiliser des logiciels
informatiques.

%\subsection{Formulation des besoins généraux}



\subsection{Exploitation, ergonomie et expérience}

Pour permettre des formations optimales, le nombre de 8 personnes par session est
recommandé, ainsi cette formation sera plus interactive. 
De plus pour la documentation devra être faite à plusieurs niveaux 
\begin{itemize}
	\item général, plusieurs documents d'ensemble seront produits pour une compréhension
	globale du système de monitoring
	\item contextuel, l'aide pourra être présent à l'intérieur même des programmes pour
	permettre aux utilisateurs en cas de doûte sur l'utilisation
\end{itemize}

%\subsection{Mise en Oeuvre}


\subsection{Organisation de la maintenance}

Ce cahier des charges sera applicable lors de toute modification du système 
impactant son fonctionnement global. Nous pouvons citer les points suivants :
\begin{itemize}
	\item le changement de logiciel ou de version majeur
	\item l'ajout de nouvelles fonctionnalités
	\item la modification de processus sur la gestion des messages
	\item le développement de nouveaux systèmes de capteurs ou de monitoring
	\item etc...
\end{itemize}

De manière général, ce document n'a pas de limite de durée de vie mais doit s'utiliser
en complément du cahier des charges d'un module spécifiquement à documenter.

\subsection{Limites}

Ce cahier des charges se limitera à présenter la mise en place de la formation 
aux utilisateurs, c'est à dire le comité du COPEVUE, les propriétaires des sites
où la solution est déployé ainsi que les sociétés d'intervention.

\pagebreak
