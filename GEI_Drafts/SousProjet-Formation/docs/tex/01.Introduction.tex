\section{Introduction}
%Présentation du projet
%Présentation du document
%Documents applicables, Document de référence
%Terminologie et abréviations

La COPEVUE a sélectionné la TEAM H4413 pour réaliser un système de monitoring pour 
surveiller les sites isolés. Ces actions ont été mise en place pour anticiper et prévenir
les risques de problèmes environnementaux ainsi que de limiter les coûts du fait
d'une non automatisation du travail. \\

Suite à la sélection de l'appel d'offre la société a identifié un ensemble de sous-projet
nécessaire pour réaliser la mise en place de ce système. Ce document est un cahier des charges
pour détailler la procédure de formation et mise en place de la documentation à destination des 
utilisateurs finaux. \\

Ce document abordera notamment les questions par rapport au planning de déploiement 
de la nouvelle solution, les acteurs intervenants dans la formation des nouveaux 
utilisateurs, les exigences nécessaires pour une formation réussie.


\subsection{Documents Applicables}

Les documents applicables sont :
\begin{itemize}
\item l'ensemble des livrables de documentation fourni aux utilisateurs finaux
\item Les cours de 3ème année et 4ème de M. Aubry sur l'ingénierie logicielle
\item le planning de déploiement de la solution
\end{itemize}

\subsection{Documents de référence}

Les documents de référence sont : 
\begin{itemize}
\item le plan d'assurance qualité du projet "système de monitoring"
\item l'ébauche de la conception détaillée
\item la synthèse de projet rédigé par l'équipe GEI
\end{itemize}

Ce cahier des charges a pour objet la mise en place d'un cahier des charges pour mettre en place la formation des utilisateurs au nouveau système. Les formations qui seront planifiés seront particulièrement adaptés aux personnes non techniques, c'est à dire les personnes intervenant actuellement sur le système actuel. \\

Ce document préparera l'accompagnement au changement tant à la fois sur la planification et le contenu des formations faites par les équipes de conseil de notre société. Ce document posera
également le socle minimal de contenu documentaire pour permettre aux acteurs d'exploiter le nouveau système.


\vfill
\pagebreak

%BUT
%Principe du logiciel
%Formulation des besoins
%Exploitation et ergonomie, expérience
%Portée, Développement, Mise en Oeuvre, Organisation de la maintenance
%Limites

