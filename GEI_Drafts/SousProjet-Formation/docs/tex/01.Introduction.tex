\section{Introduction}
%Présentation du projet
%Présentation du document
%Documents applicables, Document de référence
%Terminologie et abréviations


Ce cahier des charges a pour objet la mise en place d'un cahier des charges pour mettre en place la formation des utilisateurs au nouveau système. Les formations qui seront planifiés seront particulièrement adaptés aux personnes non techniques, c'est à dire les personnes intervenant actuellement sur le système actuel. \\

Ce document préparera l'accompagnement au changement tant à la fois sur la planification et le contenu des formations faites par les équipes de conseil de notre société. Ce document posera
également le socle minimal de contenu documentaire pour permettre aux acteurs d'exploiter le nouveau système. \\

%Acteurs
Ce document vise les formations aux acteurs suivants : 
\begin{description}
\item[le COPEVUE] qui se servira de ce système comme un outil de supervision globale des sites sous surveillance,
\item[les propriétaires] qui pourront suivre à tout moment l'état de leurs cuves à travers un outil simple, 
\item[les sociétés d'intervention] qui effectueront les missions d'interventions sur les sites.
\end{description}

La mise en place du système de monitoring va impacter les métiers de chaque acteur. Ces acteurs ne sont pas nécessairement formés à l'utilisation d'une informatique avancée.

\subsection{Organisation de la montée en compétence des équipes}
%BLABLA PAQ



\subsection{Suivi des évolutions}

Ce cahier des charges devra être révisé lors de toute modification du système impactant son fonctionnement global. Nous pouvons citer les points suivants :
\begin{itemize}
\item le changement de logiciel ou de version majeur
\item l'ajout de nouvelles fonctionnalités
\item la modification de processus sur la gestion des messages
\item le développement de nouveaux systèmes de capteurs ou de monitoring
\item etc...
\end{itemize}

\subsection{Limites}

Ce document n'a pas vocation à décrire le fonctionnement des API et des services entre eux. Ce cahier des charges va principalement décrire l'expérience utilisatrice pour les acteurs décrits dans les partis préalable. 





%BUT
%Principe du logiciel
%Formulation des besoins
%Exploitation et ergonomie, expérience
%Portée, Développement, Mise en Oeuvre, Organisation de la maintenance
%Limites

