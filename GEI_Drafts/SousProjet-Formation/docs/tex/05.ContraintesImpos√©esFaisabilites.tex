\section{Contraintes imposées, faisabilité technologique et éventuellement moyens}
%planning, organisation, communication
%complexité
%compétences, moyens et règles
%norme de documentation

Les futurs utilisateurs ne sont pas forcément formés à l'utilisation de 
logiciels informatiques avancés. Les formations devront être accessibles mais 
complètes pour avoir une vu d'ensemble du système de monitoring. \\

Les documents et formations devront être produits au moins en anglais, langue
de travail de l'union européenne. 

\subsection{Organisation}

Afin de tester la mise en place des formations : un petit groupe d'utilisateur
servira comme testeur. Ils permettront de faire des retours sur les manques et 
les besoins des formations. 

\subsection{Planning}

Pour la première mise en place des sessions de formation, il faudra mettre en 
place le planning suivant.

\vspace{0.8cm}
\begin{tabular}{|r|l|}
\hline
Semaine&Action\\
\hline
Au préalable& Mise en place de la formation\\
\hline
Au préalable& Identifier un groupe volontaire\\
\hline
Au préalable& Créer les supports\\
\hline
S1& Faire la formation au groupe volontaire\\
\hline
S1& Obtenir des retours sur la formation\\
\hline
S2& Intégrer ces retours et les analyser\\
\hline
Au plus tôt S3& Mettre en place les formations pour l'ensemble des équipes\\
\hline
\end{tabular}
\vspace{0.8cm}


\textit{Les semaines sont relatives par rapport au lancement du cycle de formation.}


\subsection{Complexité}

Le niveau de complexité sera adapté au niveau des utilisateurs. Les formations
se voudront dictatiques pour permettre à l'ensemble des utilisateurs de pouvoir
progresser.

On veillera à éviter la complexité inutile (exemple : un dirigent de la COPEVUE
n'a pas besoin de connaître la procédure de remise en marche du système embarqué). 
Néanmoins ces documents lui seront accessibles si il le souhaite.

\subsection{Moyens}

Un groupe de volontaire devra être trouvé pour tester les premières sessions de
formation. Il faut privilégier les utilsateurs qui ont participés à la mise en
oeuvre du cahier des charges pour la copevue. \\

De plus il faudra créer une communauté d'utilisateurs pour généraliser l'ensemble
des best-practises utilisées au sein de la COPEVUE. \\

Pour permettre des conditions idéales de formation, les organisateurs dédieront 
des salles spéciales aux sessions de formation pour permettre aux personnes en 
formation de bénéficier des meilleurs conditions.




\subsection{Règles}

\begin{itemize}
	\item La règle des 2 heures pour la durée maximale d'une formation
\end{itemize}

