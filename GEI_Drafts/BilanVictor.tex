\section{Victor Borges}

\subsection{Informations Générales}

Victor Borges \\
Temps passé hors séance : 20h \\
Rôle : GEI

\subsection{Déroulement du projet}

En tant que GEI, ce projet m’a permis de prendre réellement conscience de l’importance d’un bon système de communication, et surtout de la planification précise des tâches à effectuer par chacun. \\

J’ai pu regretter une certaine manque d’anticipation et de communication concernant les tâches à effectuer du projet. En effet, je trouve que la communication au sein de notre groupe était quasiment nulle. Nous savions ce que nous avions à faire, parfois tard par rapport à la date de rendu, mais nous ne savions absolument pas ce qui se passait du côté des autres membres de l’héxanôme. \\

L’hétérogénéité  de l’équipe à tous le niveaux a rendu quelques fois le travail assez difficile mais à la fin on a réussi à rendre un produit correcte et complète.

\subsection{Remarques personnelles}

L’approche qualité définie de manière très précise par notre RQ m’a obligé à plus de rigueur dans mon travail, et je considère ceci comme un atout pour les futurs projets auxquels je contribuerai, aussi bien dans ma vie scolaire que professionnelle. \\

Ce projet était également très intéressant par le travail de veille technologique qu’il a nécessité afin de répondre à chaque exigence de l’organisation COPEVUE. Cet exercice me permettra je pense d’être plus efficace et exhaustive dans les tâches similaires qui pourrait m’être confiées.


