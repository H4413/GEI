\documentclass [a4paper] {report}
\usepackage[utf8]{inputenc}
\usepackage[francais]{babel}
\usepackage[top=2cm, bottom=2cm, left=2cm, right=2cm]{geometry}
\usepackage{hyperref}
\usepackage{eurosym}
\begin {document}
\section*{Quentin Villers}

\subsection*{Informations Générales}

Quentin Villers \\
Temps passé hors séance : 12h \\
Rôle : GEI

\subsection*{Déroulement du projet}

Le projet s'est relativement bien passé. En plus de mon travail de GEI, j'ai 
participé à l'élaboration de la présentation finale et à la fiche commerciale.

\subsection*{Remarques sur le projet}
\begin{itemize}
\item Manque de visibilité sur l'ensemble du projet, en effet j'ai trouvé les
interactions faibles entre les GEI, le chef de Projet et le responsable qualité.
De plus on ne voyait pas forcément l'enchainement entre les séances.
Les principaux livrables comme le dossier d'initialisation et la PAQ n'étant pas
présentée aux GEI pendant le projet. 
\item Le sujet du projet est intéressant, il permet de monter en compétence aux GEI sur
les domaines qui leur plaisent. En effet ce sujet comprend un vaste ensemble de métiers
de l'informatique, des télécoms aux ERP en passant par les systèmes embarqués.
\item Le travail en équipe est difficile à mettre en place par rapport aux 
nombres de livrables à fournir en si peu de temps. En effet nous n'avons que peu 
de temps pour la réflexion et pour faire les choses bien. De plus la revue intermédiaire
a eu lieu trop tard. Pourquoi ne pas réfléchir à laisser une semaine de battement
entre la fin de la première phase et le début de la deuxième phase?
\end{itemize}

\subsection*{Remarques personnelles}
\begin{itemize}
\item Devant la charge importante de travail sur l'ensemble de ce projet, j'ai progressé
surtout en "focus", la capacité à travailler sur un projet en étant concentré à 100\%
sur une durée plus longue.

\end{itemize}
\end{document}
