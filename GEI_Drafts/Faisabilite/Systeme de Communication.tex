\subsection{Système de Communication à Distance}

Nous avons le choix de plusieurs technologies différentes: la radiodiffusion, GSM, 3G, WiFi, etc.

\subsection{\subsection{Transmetteurs Radio – La radiodiffusion}}

La radiodiffusion a la particularité de permettre une communication asymétrique. 
Une station de radio est une installation qui émet des ondes électromagnétiques à l'aide d'un émetteur radio et d'une antenne. 
Un poste de radio ou récepteur radio est un appareil permettant de recevoir les ondes radio, en extraire la modulation et restituer les sons sur un haut-parleur. 

Exemple d'émetteur adapté à notre but au marché: 

SOFREL BOX

\begin{figure}
\begin{center}

% sofrelbox.png

\end{center}
\end{figure}

Les transmetteurs SOFREL BOX sont spécifiquement développés pour le télécontrôle de petites installations isolées et sans énergie.
Très simples à installer et à utiliser, les SOFREL BOX constituent une solution particulièrement adaptée pour les asservissements entre un site isolé et le serveur.
Ils permettent notamment :
L’acquisition d’informations de contrôle (niveaux, compteurs,…)
La communication inter-sites vers un poste local de télégestion SOFREL S500
Doté d’un modem radio intégré pour, le transmetteur déclenche un appel spontané sur le changement vers le S500 de la station.
Hors alarme, les transmissions d’informations se font régulièrement selon une période paramétrable (toutes les 3, 5, 10 ou 15 minutes).

\subsection{\subsection{Transmetteur GSM}}

Exemple: CELLBOX

Développé pour la surveillance technique des sites dépourvus de source d’énergie et de toute liaison de communication filaire (RTC, LS/LP…), CELLBOX est un poste local de télégestion autonome communiquant par GSM (900/1800 Mhz).
CELLBOX assure l’acquisition et l’enregistrement d’informations d’alarmes, de comptages et de mesures, et effectue automatiquement différents calculs et bilans.
Etanche, CELLBOX intègre dans son boîtier une pile qui lui procure une autonomie totale de fonctionnement de plusieurs années.
Economique, simple et rapide à mettre en oeuvre, CELLBOX apporte une solution performante pour répondre à de multiples applications : sectorisation de réseaux d’eau, recherche de fuites, télégestion de sites isolés...


\subsection{\subsection{Transmetteur 3G/3G+}}

Peu adapté, car le site doit avoir un réseau 3G.

Exemple : VIGICOM VID-1500

\begin{figure}
\begin{center}

% vigicom.png

\end{center}
\end{figure}

VigiCom VID-1500 est un transmetteur  3G spécialement étudié pour équiper tout véhicule ou site isolé. C’est un équipement complet de vidéo-surveillance temps réel par transmission via le réseau cellulaire (3G) réalise une compression vidéo H264 matérielle, transmet depuis deux canaux indépendants et enregistre simultanément sur deux autres canaux séparés les vidéos en qualité DVD sur carte mémoire amovible.

Avantages:
Localisation en temps réel grâce au GPS intégré. 
Grande fiabilité, aucune pièce mécanique, enregistrement sur carte mémoire.
