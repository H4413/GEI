\subsection{capteurs}

Les capteurs dont nous auront besoin doivent satisfaire à plusieurs
contraintes principales: celle de la basse consommation d'énergie
(et donc d'autonomie), celle de la précision, de la modularité (
facilité de changement, de remplacement, de déplacement).

La société française IJINUS conçoit ainsi des capteurs qui nous
conviendraient: communicants, ils peuvent faire partie d'un
réseau de capteurs. Ils sont en outre précis, et pas seulement
pour des liquides en utilisant une mesure par imagerie acoustique.
Ils sont de plus utilisables en environnement difficile, voire agressif.
La tête de lecture est autonettoyante, permettant de ne pas avoir
à intervenir si la tête est salle.

\begin{figure}[!h]
\begin{center}

\includegraphics[width=7cm]{\PICSPATH{}ijinus}
\caption{Capteurs proposés par IJINUS}
\end{center}
\end{figure}

Ce n'est pas la seule entreprise à proposer ce genre de matériel (comme la société SICK avec sa série LFVXXX), nous
n'aurons donc a priori pas trop de mal à trouver notre bonheur au meilleur
prix.

%NOTE DE RELECTURE : attention dernier paragraphe. Être un peu plus précis.
%Étudier éventuellement l'offre d'une autre entreprise. C'est bien beau de
%mettre \% Tableau Capteurs, il le faudrait ce tableau ;-)
