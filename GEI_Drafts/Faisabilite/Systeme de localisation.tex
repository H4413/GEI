\subsection{Système de localisation}

\subsection{\subsection{GPS – Global Positioning System}}

Le GPS fonctionne grâce au calcul de la distance qui sépare un récepteur GPS et plusieurs satellites. Les informations nécessaires au calcul de la position des 31 satellites étant transmise régulièrement au récepteur, celui-ci peut, grâce à la connaissance de la distance qui le sépare des satellites, connaître ses coordonnées.

Nous pouvons aisement utiliser un module GPS embarqué pour savoir à tout momment les positions exacte des sites.

Quelques modules de GPS embarqué:

Exemple 1 : Module SKM53 avec antenna

\begin{figure}
\begin{center}

% modulegps.png

\end{center}
\end{figure

La série SkyNav SKM53 avec antenne permet une navigation de haute performance dans les applications les plus rigoureuses.
Il est basé sur les caractéristiques de haute performance de la MediaTek 3327/3329 architecture mono-puce, sa sensibilité de suivi 165dBm étend la couverture de positionnement. Il est la solution la plus simple et commode pour être embarqué dans un appareil portable.

Prix: 32 euros

Exemple 2: Embedded GPS/GALILEO PCI Express Mini Card
Cette carte peut être relié directement à l'ordinateur par une entré PCI.
Prix: environ 50 euros.

