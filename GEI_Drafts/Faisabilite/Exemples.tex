\section{Solutions techniques existantes}

Ce que nous voulons faire peut s'appuyer sur du matériel et 
du logiciel déjà existant, qu'il faudra peu ou pas adapter pour
satisfaire les exigences de ce projet. Voici des exemples de
capteurs et serveurs qui pourront nous servir de base de
travail.

\subsection{capteurs}

Les capteurs dont nous auront besoin doivent satisfaire à plusieurs
contraintes principales: celle de la basse consommation d'énergie
(et donc d'autonomie), celle de la précision, de la modularité (
facilité de changement, de remplacement, de déplacement).

La société française IJINUS conçoit ainsi des capteurs qui nous
conviendraient: communicants, ils peuvent faire partie d'un
réseau de capteurs. Ils sont en outre précis, et pas seulement
pour des liquides en utilisant une mesure par imagerie acoustique.
Ils sont de plus utilisables en environnement difficile, voire agressif.
La tête de lecture est autonettoyante, permettant de ne pas avoir
à intervenir si la tête est salle.

\begin{figure}
\begin{center}

% Tableau capteurs

\end{center}
\end{figure}


\subsection{serveurs}

L'entreprise IJINUS propose un service de gestion à distance et en temps
réel pour les capteurs, et fournissent également un service de
réapprovisionnement par camions.
Sur leur interface, il est possible de suivre en direct l'évolution des
mesures des capteurs, ainsi que de consulter l'historique de ces mesures.
