\subsection{Serveurs}

    \subsubsection{Exigences}
    Il nous faut un ensemble se serveurs capables de répondre à un grand
    nombre de requêtes, et ce en permanence. Il doit également être capable
    de conserver les traces des connections et données reçues.

    \subsubsection{Solutions}
        \begin{enumerate}
            \item Monter nos serveurs
            
            La première solution est de monter nos propres serveurs nous
-mêmes.
            L'avantage est que nous avons les locaux pour héberger une telle solution, et qu'il sera possible de faire évoluer le parc en fonction des besions.
    
            Les inconvénients sont cependant nonbreux: il faudrait embaucher une équipe dédiée à la gestion de ces serveurs, mais également les entretenir,
            les faire évoluer. De plus, ils sont très onéreux à l'achat.

            \item Louer des serveurs
            Une seconde solution consiste à louer des serveurs, dans un \textsl{datacenter} par exemple.

            Les avantages sont nombreux: nous n'aurions pas à engager une équipe pour
            nous en occuper (physiquement), nous aurions une assurance de service
            (pouvant aller jusqu'à 99.995\% de temps en ligne) et nos données seraient
            en lieu sûr (les données sont souvent copiées en plusieurs endroits, en cas de sinistre). Là encore, il serait possible de trouver une offre adaptée à nos besoins. Nous n'aurions pas à supporter le coût du matériel et de la maintenance, tout en ayant toujours un servuce assuré.

            Un inconvénient peut être la dépendance au service, mais ça n'en est pas vraiment un, vu le nombre de fournisseurs sur le marché.
    \subsection{conclusion}
    Il semblerait qu'il soit préférable de louer les services d'un datacenter, du moins pendant la phase d'exploitation.
% Chercher offres 
