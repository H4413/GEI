%!TEX encoding = UTF-8 Unicode
\documentclass [a4paper] {report}
\usepackage[utf8]{inputenc}
\usepackage[francais]{babel}
\usepackage[top=2cm, bottom=2cm, left=2cm, right=2cm]{geometry}
\usepackage{hyperref}
\usepackage{eurosym}
\begin {document}
\title{GEI}
\date{Janvier-2011}
\maketitle

\chapter{Gestion de l'Energie}
\section{Exigences}
\begin{itemize}

\item Autonomie: Les stations sont souvent situes dans des endroits tr\`es isol\'{e}es et d'acc\`es compliqu\'e. Alors une solution c'est que les ressources naturelles peuvent contribuer  à l'autonomie \'energ\'{e}tique des stations et il est \'{e}vident que la consommation \'{e}lectrique doit \^etre minimale.

\smallskip \item Robustesse et Fiabilit\'{e}: Même en cas de probl\`eme environnemental l'approvisionnement de l'\'{e}nergie \`a la station doit \^etre assur\'{e} sans intervention humaine.

\smallskip \item Protection de l environnement: les stations peuvent se trouver dans des espaces prot\'{e}g\'{e}s donc il faut mettre en place de l'\'{e}nergie non polluant et renouvelable car elle utilise des flux d'\'{e}nergie naturelle(soleil, vent, eau,croissance v\'{e}g\'{e}tale,....).
\end{itemize}

\section{Alternatives}
\subsection{Energie Solaire Photovolta\"ique}
\begin{enumerate}
\item \textbf{Avantages :}
Les syst\`emes solaires photovoltaïques sont simples et rapides à installer, l'investissement  et rendement sont pr\'{e}visibles à long terme.
Elle r\'{e}pond aux exigences de robustesse et fiabilit\'{e} car est tolérante aux pannes et n\'{e}cessite tr\`es peu de maintenance. 
Elle est exploitable pratiquement partout, la lumi\`ere du soleil \'{e}tant disponible dans le monde entier. 
Aussi, cette production d'\'{e}nergie ne provoque aucun perturbation pour l environnement.

\item \textbf{Risques et Inconv\'enients :}
Ces installations n\'{e}cessitant un apport en soleil, les pays proches des pôles, moins expos\'{e}s aux rayons solaires, sont tr\`es peu rentables.
\end{enumerate}

\subsection{Energie \'{e}olienne}
\begin{enumerate}
\item \textbf{Avantages :}
L'\'{e}nergie \'{e}olienne est renouvelable et est id\'{e}ale parce que il s'agit d une forme d'\'{e}nergie ind\'{e}finiment durable.
Elle ne produit pas de d\'{e}chets toxiques ou radioactifs. 
Son temps de fonctionnement est environ 20 ans.

\item \textbf{Risques et Inconv\'enients :}
L'\'{e}nergie \'{e}olienne varie dans le temps.  Elle tourne uniquement s il y a du vent.
La localisation des \'{e}oliennes est d\'{e}pendante de la ressource(le vent) et on ne peut pas les implanter n importe ou. 
Impact sur la faune, les \'{e}tudes ont constat\'{e} que des \'{e}oliennes \'{e}taient responsable de la mort de quelques oiseaux.
\end{enumerate}


\subsection{Batterie d Accumulateurs}
\begin{enumerate}
\item \textbf{Avantages :}
Pour une alimentation sans interruption, les batteries stockent l'energie permettant de supplier de quelques minutes à quelques heures.
Il existe les batteries rechargeables qui sont orient\'{e}es pour un fonctionnement avec des panneaux photovoltaïques ou  \'{e}oliennes(dur\'{e}e de vie augment\'{e}, propriét\'{e} anticorrosion grâces à des plaques positives \'{e}paisses, recyclable)

\item \textbf{Inconv\'{e}nients :}
Sont produits lourds donc le frais de port peuvent s av\'{e}rer cons\'{e}quents.
Polluant

\end{enumerate}

\section{Choix de Solution}
Apres un analyse des exigences du syst\`eme nous proposons la combinaison de l utilisation des batteries pour le stockage, et pour la production une \'{e}nergie renouvelable pour satisfaire l autonomie \'{e}nerg\'{e}tique.


\bigskip Auteur: Karen ABANTO\\
Positionnement: Dossier Faisabilit\'{e}(gestion de Energie)\\
Objectif: Chercher solution pour la production de l'\'{e}nergie dans un endroit isol\'{e}e

\end{document}