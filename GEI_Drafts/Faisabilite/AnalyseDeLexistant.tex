\section{Analyse de l'existant}

\subsection{Anaylse du métier}

La part métier de l'existant est relativement peut importante. Pour l'utilisation du système de monitoring, nous mentionnerons néanmoins le travail de planification effectué par les compagnies d'intervention et le travail de surveillance du propriétaire. Pour la dernière nous ne pouvons pas considérer que cela constitue un travail en soit.

\subsection{Analyse des savoir-faire et des processus}

\subsubsection{Suivi des sites}

Le propriétaire est en charge de suivre ses sites. Il contrôle le niveau de ses réservoirs et lorsqu'il juge l'intervention nécessaire, il appelle la société d'intervention en charge de son réservoir. Il faut noter qu'il n'y a pas de critères de taux de remplissage et de prévisibilité d'une demande d'intervention.

\subsubsection{Planification}

La planification est organisation par la société d'intervention. Celle ci est organisé en fonction des demandes des propriétaires. Les principes lacunes par rapport à ce système sont : 
\begin{itemize}
\item Manque de suivi du remplissage pour anticiper les interventions
\item Absence de seuil critique pour être sur d'un remplissage optimal des camions
\end{itemize}

\subsection{Analyse du matériel utilisé}

Les matériels utilisés qui sont utilisés comme les camions pour le transport du stock des réservoirs seraient encore utilisés. La vérification de l'état des réservoirs se fait de manière artisanale par le propriétaire du lieu et les demandes d'interventions doivent se faire par une demande spéciale du propriétaire par un appel téléphonique ou un système de messagerie. La planification des interventions par les entreprises se fait de manière artisanale en fonction des demandes.




