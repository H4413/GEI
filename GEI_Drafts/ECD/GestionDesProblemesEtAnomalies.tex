\section{Gestion des problèmes et des anomalies}

Nous allons classer les problèmes selon leur effet visible pour l'utilisateur à 
partir du serveur central. 

\subsection{Panne prévisible}

\subsubsection{Batterie déchargée}

Cette panne est prévisible. En effet en suivant le cycle de décharge 
du système embarqué nous pouvons prévoir le déchargement du système embarqué. 

Pour éviter cette panne, il faudra prévoir une intervention pour changer la batterie
du système embarqué. De plus un système d'alerte sera mis en place pour informer
 les propriétaires et les sociétés d'intervention de la criticité de la charge 
de la batterie. 

\subsection{Panne non prévisible et communication avec le serveur central}

\subsection{Valeur aberrante issue d'un capteur}

En cas de défaillance d'un capteur, celui-ci peut fournir des valeurs abérantes 
ou plus de valeur.Le système embarqué pourra reconnaître l'absence de valeur mais
 pas le manque de sens du valeur. Néanmoins le serveur central pourra déceler des
 valeurs particulières selon des règles de traitement prédéfinies. 

Cette panne sera traité par un changement de capteur.

\subsection{Problème avec l'antenne bluetooth}

Cette panne ne sera visible que lors d'une connexion locale. Elle n'est pas 
détectable en ligne. Pour réparer ce problème il faudra change le module bluetooth.

\subsection{Pannes non prévisibles sans communication avec le serveur central}

L'ensemble de ces pannes ne sont pas prévisibles et entrainent une absence de 
communication avec le serveur. Il peut être difficile de trouver la source de ces 
pannes sans se déplacer sur les sites.

\subsubsection{Problème du module GSM}

Cette panne n'est rare et non prévisible. 

\textbf{Diagnostic} \\

L'opérateur pourra se connecter avec son client mobile au système embarqué, 
ce qui prouvera que le système embarqué fonctionne bien.
De plus il captera l'antenne GSM avec son téléphone portable. 

\textbf{Recouvrement} \\

L'opérateur changera le module GSM.

\subsubsection{Problème de relais GSM ou de relais Internet}

Ces pannes sont jugées comme peu probable et mineure. Ces pannes sont généralement 
traitées dans la journée par les opérateurs.

\subsubsection{Défaillance du serveur central}

Ces pannes sont jugées comme peu probable et mineure. 
L'hébergement chez un hôte profesionnel du cloud computing va nous permettre de 
garantir la disponibilité du serveur à tout instant. De plus les données sont 
protégées car stockées sur plusieurs sites et de manière redondante.

\subsubsection{Catastrophe naturelle, Dégats des eaux,  ...}

Dans des milieux soumis fortemment aux contraintes naturelles, cette panne est jugée crédible.
L'opérateur pourra repérer que le système de fonctionne plus, malgré avoir changé la batterie. 
Il faudra alors procéder au remplacement du système embarqué.



