\section{Architecture Informatique et Matérielle}

\subsection{Architecture Générale}
Afin de limiter le coût de développement d'un logiciel spécifique. 
Nous allons utiliser le progiciel SAP. Celui-ci sera paramétré pour
communiquer avec les différents interlocuteurs. 

\subsection{Systèmes embarqués}

Les systèmes embarqués tourneront sous un système d'exploitation Linux. 

les systèmes auront les composants suivants : 
\begin{itemize}
\item{}
\item{}
\item{}
\item{}
\end{itemize}


%définir le matos

\subsection{Clients mobiles}

Les clients mobiles utilisés pour communiquer localement avec les systèmes embarqués auront besoin des spécifications suivantes : 

- Accès mobile par un accès bluetooth
- Navigateur internet pour accéder au serveur web du système embarqué

Le principal avantage de la connexion par un navigateur internet au système embarqué et que le client mobile n'aura pas à installer d'application tierce.

\subsection{Serveurs}

Les serveurs seront hébergés sur le cloud. Nous pouvons choisir l'hébergeur GoGrid (fournisseur IAAS) 
qui propose des solutions adaptées en fonction de l'utilisation du serveur.

L'utilisation d'un hébergeur va permettre de limiter les coûts d'utilisation du serveur. 
En effet l'activité des serveurs est à la fois saisonale 
(\textit{il est plus important de surveiller l'activité des sites méditérannéens lors de la saison des incendies}) et
périodique sur un délai du jour. En effet les échanges auront lieu à une heure donnée et l'utilisation du serveur par les acteurs tiers aura lieu pendant les heures de bureaux (de 8:00 à 18:00)

L'ensemble de ces facteurs nous font choisir l'hébergement par GoGrid puisqu'ils permettent d'attribuer les ressources dynamiquement en fonction des besoins. De plus le COPEVUE n'aura a payé que l'utilisation des ressources. De plus ces hébergeurs sont spécialistes dans la sécurisation des données et proposent de multiples redondances. 

Dans un premier temps, nous choissirons le plan "Pay as You Go" qui sera idéal pour soutenir les communications avec une centaine de sites pilotes. 
Le prix estimé est de 100$ par mois, pour une centaine de sites.

http://www.gogrid.com/cloud-hosting/cloud-hosting-pricing.php

\subsection{}
