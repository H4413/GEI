\section{Architecture Informatique et Matérielle}

\subsection{Architecture Générale}
Afin de limiter le coût de développement d'un logiciel spécifique. 
Nous allons utiliser le progiciel SAP. Celui-ci sera paramétré pour
communiquer avec les différents interlocuteurs. 

\subsection{Systèmes embarqués}

Les systèmes embarqués tourneront sous un système d'exploitation Linux ou VxWorks. 

Pour permettre une gestion optimale de l'énergie, le système embarqué sera actif
uniquement pendant un cours laps de temps par jour: le système de prise de mesure
autant de fois qu'il faut prendre des mesures dans la journée, le système de
transmission une fois par jour pour transmettre les données collectées pendant
les dernières 24 heures, et éventuellement pour être configurés.
Tout cela doit être règlable, au cas ou l'on veuille par exemple plusieurs transmissions
par jour ou au contraire mois d'une transmission par jour, et pour régler l'intervalle de
prises de mesures.

Un technicien peut manuellement mettre en route le dispositif sur place pour
le configurer et consulter ses données.

Quand le système ne sera pas en fonctionnement, il sera en mode veille, lui
permettant d'utiliser le moins d'énergie possible.

%définir le matos

\subsection{Serveurs}

Les serveurs seront hébergés sur le cloud. Nous pouvons choisir l'hébergeur GoGrid 
qui propose des solutions adaptées en fonction de l'utilisation du serveur.

L'utilisation de cet hébergeur va permettre de limiter les coûts d'utilisation du serveur. 
En effet l'activité des serveurs est à la fois saisonale

%Lien http://www.gogrid.com/
