\section{Architecture Applicative}

\section{Systèmes embarqués}

Les systèmes embarqués fonctionneront sous une distribution de Linux. Le
coût de license sera ainsi nul. 

Les systèmes embarqués pourront héberger un serveur web qui ne sera
accessible que localement par les propriétaires ou les sociétés
d'interventions. Cette interface comportera seulement quelques pages avec
notamment des logs et des options pour le paramétrage. Cette application
web ressemblera à une interface de gestion de box.

En cas de problème plus important on pourra se connecter via une console de
type SSH.

\section{Serveurs Centraux}

Le serveur central comprendra une base de donnnées sur laquelle on fera les
communications avec le système embarqué. Nous pourrons aussi y placer des
applications de mise à jour de la version des logiciels de système
embarqué. 

Le système de gestion du monitoring et des gestions des interventions est
OpenERP. Un ERP est parfaitement adapté à la gestion de ce système de
surveillance. De plus on peut l'associer avec des systèmes de Business
Intelligence comme Pentaho ou Jasper pour faire du reporting. Ainsi selon
les profils, on pourra générer des rapports facilement et dynamiquement en
fonction des besoins.  De plus les compagnies d'intervention pourront
traiter leurs interventions via ce système et gérer éventuellement la
facturation (hors cadre de l'appel d'offre).

L'utilisation d'un ERP open source va donc limiter les coûts logiciels
(seulement du paramétrage) et limiter la taille du développement
spécifique. De plus il s'inscrit dans une démarche de progicielisation qui
permettra d'augmenter simplement le nombre et le type de capteurs ainsi que
de diversifier les intervenants dans ce système.

Pour aller plus loin : 
\begin{itemize}
\item Jasper Soft outil de reporting et de BI http://www.jaspersoft.com/
\item OpenERP, l'ERP libre de référence http://www.openerp.com/
\end{itemize}

\section{Utilisation distante}

L'utilisation à distance va pouvoir se faire via un navigateur web. OpenERP
possède directement une interface web qui est paramétrable en fonction des
profils.

En se connectant à l'ERP les acteurs de ce système pourront suivre la
situation de leurs cuves.

Selon les besoins des utilisateurs, les outils de reporting feront des
rapports aux décideurs sur la fréquence des besoins établis. (Exemple: le
COPEVUE veut connaître le nombre d'interventions faites par pays tous les
mois). Ces reportings seront accessibles envoyés automatiquement par email
ou disponible via l'ERP. 
