\section{Contexte}

\subsection{Cadre de l'�tude et objectifs}

Cette �tude intervient suite � l'appel d'offre de la soci�t� COPEVUE. Il s'agit
de r�aliser un syst�me pour la surveillance � distance de sites isol�s.

\subsection{Critique de l'existant}

La gestion des sites se fait de mani�re totalement artisanale. Le propri�taire
d�cide lorsqu'il consid�re sa cave plein de demander une intervention.
A partir de l�, le remplissage des camions est plut�t al�atoire et il y a donc
une perte � ce niveau. La planification en souffre elle aussi car il devient
difficile de planifier des trajets optimaux pour le remplissage d'un camion.

\subsection{Axes de progr�s}

Les axes de progr�s principaux sont aux niveaux suivants :

\begin{description}
\item[Communication globale] Le passage � un syst�me automatis� va permettre 
une meilleure vision globale ainsi qu'une communication plus rapide.
\item[Tra�abilit�] Le suivi �tant quasiment inexistant la mise en place d'un ERP 
va permettre une grande tra�abilit�.
\item[Automatisation] On passe d'un syst�me artisinal � un syst�me automatis�.
\item[Autonomie des syst�mes] L'autonomie des sites sera un des premiers 
crit�res de choix.
\item[Fiabilit� et Robustesse] La plupart des pannes sont auto-g�r�es par le 
syst�me. Les pannes plus graves seront trait�es dans les plus brefs d�lais par
r�ception d'alertes.
\end{description}
