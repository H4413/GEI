% Projet Ingéniérie - Dossier d'initialisation - Clément Geiger

\documentclass[twoside]{article}
\usepackage{hyperref}


\usepackage{graphicx}
\usepackage{subfig}
\usepackage{placeins}


% Colorfull Text
\usepackage{xcolor}


% Language settings:
\usepackage[francais]{babel}

\usepackage[T1]{fontenc}


% Tables
\usepackage{array}
\usepackage{longtable}


% Hyperrefferences  
\usepackage{hyperref}


% Font settings:
\usepackage{charter}


	itle{Projet Ingéniérie - Dossier d'initialisation}
uthor{Clément Geiger}
% Page layout settings
\usepackage{geometry}
\geometry{
	a4paper,  % 21 x 29,7 cm
	body={160mm,240mm},
	left=30mm, 
	top=25mm,
	headheight=7mm, 
	headsep=4mm,
	marginparsep=4mm,
	marginparwidth=27mm
}


% Spacing:
\usepackage{setspace}


% Headers and footers:
\usepackage{fancyhdr}
\pagestyle{fancy}
          ancyhf{}
          ancyfoot[LE,RO]{	extcolor[gray]{0.3}{	hepage}}
          % Rulers width
          
enewcommand{ootrulewidth}{.3pt}
          
enewcommand{\headrulewidth}{.3pt}
ancyfoot[LO,RE]{	extcolor[gray]{0.3}{Clément Geiger}}
ancyfoot[CO,CE]{	extcolor[gray]{0.3}{Projet Ingéniérie - Dossier d'initialisation}}


% Vars & functs

ewcommand\PIXPATH{./docs/pics}

ewcommand\SRCPATH{./docs/src}

ewcommand\Object{}

enewcommand{\labelitemi}{$\diamond$}

enewcommand{\labelenumi}{(lph{enumi})}


% Begining of the document
egin{document}

	%Including all the files:

    % Fichier ./docs/tex/0.a.premiere_page.tex

% Front Page 

% Title:
\maketitle

\thispagestyle{empty}

\hfill\\
\vfill

% Picture

\begin{center}
    \includegraphics[width=3cm]{\PIXPATH/insa}\hfill\\
    \includegraphics[width=5cm]{\PIXPATH/frontPage}
\end{center}

\section*{Objet}
\Object

\pagebreak

    % Fichier ./docs/tex/0.b.suivi.tex

% Suivi du document

% Modifications
\section*{Modifications du document}

\begin{center}
\begin{longtable}{|m{14mm}||m{36mm}|m{36mm}|m{60mm}|}
\hline
Version & Auteur & Date & Modification\endhead \hline
% Version
0
& % Auteur
\author
& % Date
This a date!
& % Modification
Création
\\\hline
% Version

& % Auteur

& % Date

& % Modification

\\\hline
\end{longtable}
\end{center}

% Validations

\section*{Vérifications et validations du document}

\begin{center}
\begin{longtable}{|m{15mm}|m{36mm}|m{36mm}|m{60mm}|}
\hline
 & Responsable & Date & Remarques\endhead \hline
% Validé/vérifié par

& % Responsable

& % Date

& % Remarques

\\\hline
% Validé/vérifié par

& % Responsable

& % Date

& % Remarques

\\\hline
\end{longtable}
\end{center}

\pagebreak

    % Fichier ./docs/tex/0.c.toc.tex

% Table of contents 
\tableofcontents
\vfill
\pagebreak

    % Fichier ./docs/tex/00.premiere_page.tex

% Front Page 

% Title:
\maketitle

\thispagestyle{empty}

\vfill

% Picture
\begin{center}
    \includegraphics[scale=0.50]{\PIXPATH/frontPage}
\end{center}

\pagebreak

% Table of contents 
\tableofcontents
\vfill
\pagebreak

    % Fichier ./docs/tex/01.intro.tex

\section{Introduction}

Avant toute chose, quelques abréviations utilisées dans ce document :

\begin{description}
\item[CdP] : Chef de Projet
\item[GEI] : Groupe d'Etude Informatique
\item[RQ] : Responsable qualité
\end{description}

Le dossier d'initialisation du projet GEI. Document rédigé par le CdP avant le 
début du projet et tenu à jour par ce dernier tout au long du projet. Il
contient diverses informations nécessaire à la bonne marche du projet telles que
la liste des livrables et des procédures, ou façon dont sera menée le projet.
    % Fichier ./docs/tex/02.contexte.tex

\section{Contexte du document}
Ce document est r
    % Fichier ./docs/tex/03.doc_reference.tex

\section{Documents de r
    % Fichier ./docs/tex/04.rappel_probleme.tex

\section{Rappel du probl
    % Fichier ./docs/tex/05.contraintes_generales.tex

\section{Contraintes g
    % Fichier ./docs/tex/06.organisation_travail.tex

\section{Organisation du travail}

Dans cette partie on d
    % Fichier ./docs/tex/07.liste_livrables.tex

\section{Liste des livrables attendus}

On trouvera ici une liste exhaustive des livrables attendus durant le projet.
Les livrables peuvent 
    % Fichier ./docs/tex/08.organigramme_taches.tex

\section{Organigramme des t
    % Fichier ./docs/tex/09.0.modalites_suivi.tex

\section{Modalit
    % Fichier ./docs/tex/09.1.gestion_risques.tex

\section{Gestion des risques}

\subsection{Risques concernant l'application du projet}
\subsection{Risques propres au projet}

    % Fichier ./docs/tex/09.2.conclusion.tex

\section{Conclusion}

Terrain.
    % Fichier ./docs/tex/09.3.annexes.tex


% The end
\end{document}

