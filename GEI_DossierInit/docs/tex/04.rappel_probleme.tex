\section{Rappel du probl�me}

Le Comit� pour la Protection de l'EnVironnement de l'UE (COPEVUE) est en
charge de surveiller un vaste nombre de site en Europe. Ces sites
surveill�s peuvent �tre tr�s diff�rents, des zones m�diterran�ennes
subissant de nombreux incendies pendant les �t�s aux r�gions nordiques
difficiles d'acc�s pendant l'hiver. 

Ces sites n�cessitent une action r�guli�re de l'homme pour des actions
diverses comme du pompage, de l'entretien o� encore des �tudes sur la faune
et la flore. Pour permettre une surveillance des sites plus efficace des
sites, le COPEVUE souhaite mettre en place un suivi � distance (monitoring)
pour leurs sites pour : 

\begin{itemize}
\item Surveiller en temps r�el l'ensemble des sites et r�duire les risques environnementaux
\item Centraliser les informations pour un meilleur suivi
\item Optimiser les actions effectu�es pour les sites pour r�duire les co�ts
\end{itemize}

Les domaines d'exigences se situent � plusieurs niveaux : 

\begin{itemize}
\item l'autonomie du syst�me embarqu�
\item les communications du syst�me embarqu� avec le syst�me central
\item la surveillance des environnements avec les capteurs et les interactions locales
\item la qualit� des interfaces utilisateurs
\end{itemize}
