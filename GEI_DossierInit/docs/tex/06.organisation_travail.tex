\section{Organisation du travail}

Dans cette partie on d�taillera les r�les des diff�rents membres de l'�quipe
menant � bien le projet.


\subsection{Chef de Projet (Cl�ment Geiger)}

Coordinateur du projet et r�f�rent aupr�s des clients.


\subsection{Responsable Qualit� (Hugo Pastore de Cristofaro)}

Responsable de la qualit� des livrables rendus aux clients et garant du respect
des proc�dures de r�daction et Bonnes Pratiques qu'il aura pris soin de mettre
en place, bien entendu. Il est conseill� au RQ de se mettre d'accord avec le RQ
du Projet Longue Dur�e afin de mettre en commun les �l�ments pouvant �tre
r�utilis�s d'un projet � l'autre.


\subsection{GEI}

\begin{itemize}
\item Karen Abanto
\item Victor Borges
\item Rapha�l Liz�
\item Quentin Villers
\end{itemize}

\hfill\\

Chevilles ouvri�res du projet, ils font ce que leur demande le CdP tout en
respectant les contraintes impos�es par le RQ.
