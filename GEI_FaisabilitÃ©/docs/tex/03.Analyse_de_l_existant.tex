\section{Analyse de l'existant}

\subsection{Analyse du métier}

La part métier de l'existant est relativement peu importante. Pour l'utilisation du système de monitoring, nous mentionnerons néanmoins le travail de planification effectué par les compagnies d'intervention et le travail de surveillance du propriétaire. Pour la dernière nous ne pouvons pas considérer que cela constitue un métier en soit.

\subsection{Analyse des savoir-faire et des processus}

\subsubsection{Suivi des sites}

Le propriétaire est en charge de suivre ses sites. Il contrôle le niveau de ses réservoirs et lorsqu'il juge l'intervention nécessaire, il appelle la société d'intervention en charge de son réservoir. Il faut noter qu'il n'y a pas de critères de taux de remplissage et de prévisibilité d'une demande d'intervention, le propriétaire appelle lorsqu'il juge que sa cave est pleine.
Il n'y a pas d'autres communications avec les sociétés d'interventions.


\subsubsection{Planification}

La planification est organisé par la société d'intervention. Celle ci est organisée en fonction des demandes des propriétaires. Les principales lacunes par rapport à ces planifications sont : 
\begin{itemize}
\item Manque de suivi du remplissage pour anticiper les interventions
\item Absence de seuil critique pour être sûr d'un remplissage optimal des camions
\item Peu d'optimisation des trajets pour remplir au maximum les camions
\end{itemize}

\subsubsection{Intervention}

Les interventions sont faites par les sociétés. Celles-ci ont lieu après la planification issue d'un appel de propriétaire. Elles ne sont pas planifiées ni optimisées. Dans la majorité des cas, les camions ressortent non plein ou non vide.

\subsection{Analyse du matériel utilisé}

Les matériels utilisés comme les camions pour le transport du stock des réservoirs seraient toujours utilisés. La vérification de l'état des réservoirs se fait de manière artisanale par le propriétaire du lieu et les demandes d'interventions doivent se faire par le propriétaire par un appel téléphonique ou un système de messagerie. La planification des interventions par les entreprises se fait de manière artisanale en fonction des demandes.



%NOTE DE RELECTURE : il fallait parler de l'existant au niveau "système de
%monitoring" et non de l'organisation préexistante dans notre cas précis.
%Ton travail sera néanmoins utile dans le cadre de la présentation du
%contexte du projet ;-)
\vfill
\pagebreak
