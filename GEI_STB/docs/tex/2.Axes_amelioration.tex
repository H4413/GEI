\section{Axes d'améliorations retenus}

Nous avons classé les axes d'améliorations principaux selon deux catégories, les principaux sur lesquels devront se baser les principaux fondements du futur système. Les axes jugés comme marginaux sont à prendre en compte mais le plus souvent ils découlent naturellement des axes principaux. 

\subsection{Axes d'améliorations principaux}

\begin{description}

	\item[Communication globale]\hfill \\

	Actuellement le suivi des sites n'est pas régulier et reste à la
	discretion des propriétaires de site, Ces interventions ne sont ni
	régulières ni prévisibles. Pour améliorer ce point, le système
	devra être en communication régulière (au moins tous les jours)
	avec les acteurs concernés (COPEVUE, Propriétaires, Compagnies
	d'interventions). Cette communication devra pouvoir se faire dans
	les deux sens, à la fois pour faire remonter les informations du
	capteur vers les utilisateurs et de ces derniers vers ses capteurs
	pour modifier le paramétrage.\\ 

	\item[Tracabilité]\hfill\\

	Actuellement les informations ne sont pas forcément classées,
	triées ou archivées. Il n'y a pas de centralisation des données. La
	mise en place d'un système de monitoring va permettre de receuillir
	des informations sur de longues durées afin d'organiser une
	prévention plus efficace, de définir les responsabilités en cas
	d'accident et d'améliorer les interventions des compagnies
	d'interventions.\\

	\item[Automatisation]\hfill\\
	
	Dans le même esprit que la communication globale, aucune
	automatisation des tâches n'est encore possible avec le système
	existant. L'un des buts de ce système sera de mettre en place le
	relevé des informations par des capteurs autonomes et d'éviter ce
	travail fastidieux aux propriétaires. Ces relevés vont devoir se
	faire sans intervention à aucun niveau de l'homme. Cette
	automatisation devra se faire sur plusieurs niveaux: de la source à
	l'analyse (traitements).\\ 
	
	\item[Autonomie des systèmes]\hfill\\

	Les environnements étant difficiles d'accès, une attention
	particulière sera mise sur l'autonomie des sites. Ces sites devant
	être autonome électriquement dans leurs relevés de mesures et leurs
	communications avec l'extérieur. On pourra mettre en place
	notamment des politiques de gestion de l'énergie pour les systèmes
	embarqués et les capteurs. Le but de cet axe d'amélioration est de
	rendre les sites indépendants, pour éviter des interventions
	inutiles aux acteurs du système.\\

	\item[Fiabilité et Robustesse]\hfill\\

	Les systèmes devront être fiables et ne nécessitent pas
	d'intervention de l'homme en cas de problèmes mineurs. Pour assurer
	une prévention de tous les instants, il faudra minimiser les
	conséquences de défaillances mineures.\\

\end{description}


\subsection{Axes d'amélioration marginaux}

\begin{description}
	\item[Prévention]\hfill\\

	Actuellement il n'existe aucune prévention par le système mis en
	place par la COPEVUE. Cette prévention générera des économies en
	limitant les désastres écologiques. Néanmoins la prévention ne
	constitue pas un axe essentiel de ce système de monitoring, il est
	marginal puisqu'il découle des axes principaux. En réalisant
	l'ensemble de ces axes principaux (communication régulière,
	automatisation et autonomie...) il sera facile de mettre en place
	une politique de prévention. La prévention est donc la conséquence
	du fonctionnement de ces axes d'améliorations.

	\item[Planification]\hfill\\

	Sans visibilité sur le long terme pour les entreprises d'entretien,
	le système devra proposer la mise en place d'une planification. Sur
	le même principe que la prévention, une planification sera permise
	grâce à une communication créée entre les systèmes embarqués et le
	serveur central et une analyse plus fine des données qui ont reçu
	cette tracabilité.

	\item[Optimisation des coûts]\hfill\\

	La réduction des coûts est certe un objectif mais il n'est pas en
	soit un axe d'amélioration principal. Cette économie sera permise
	grace à une prévention accrue qui permettra d'éviter de nombreux
	désastres écologiques. Néanmoins la mise en place de ce système de
	monitoring ne garantie pas la réduction de ces coûts.

\end{description}
