\section{Description des exigences non fonctionnelles du futur système}

Pour ce projet, nous avons identifié les exigences non fonctionnelles suivantes à partir des axes d’amélioration:

\begin{itemize}

\item Intégration de l’existant \\
Le nouveau système doit pouvoir s’intégrer dans les structures déjà mises en place pour la surveillance, les camionneurs.

\item Robustesse et fiabilité \\
Le système doit toujours être fiable et sûr. Dans un cas de redémarrage intempestif (pour cause d’une alimentation électrique défaillante par exemple), il doit revenir dans un état stable.

\item Généricité \\
Le système doit être générique et adaptable à d’autres situations, car cette 
même problématique est trouvée assez fréquemment. Il faut alors pouvoir décliner 
ce système à coût minimal pour d’autres applications de surveillance. Pour cela 
les modules, les capteurs, l’interface, etc… doivent être paramétrables.

\item Evolutivité et Maintenabilité \\
Le système doit être aisément modifiable pour en améliorer les performances et 
les fonctionnalités. Des évolutions dimensionnelle, fonctionnelle et matérielle.

\item Traçabilité \\
Le système doit garder une trace de toute activité effectuée pendant 2 ans. 
Cette trace permettra de revoir une éventuelle erreur et d’analyser les données, 
elle doit donc être enregistrée dans un serveur fiable.

\item Limitations Techniques \\
Le système va utiliser des technologies qu'il ne maîtrise pas totalement (GPS, GPRS). 
Il faut donc adapter le système aux différentes situations qui peuvent survenir.

\item Réutilisation \\
Pour des raisons de coûts évidentes, il faut réutiliser toutes les parties de 
l'ancien système adaptables dans le nouveau système. En même temps, la conception 
du nouveau système devra être orientée pour la réutilisation (pour les autres systèmes à venir).

\item Ergonomie \\
L'interface homme-machine doit prendre en compte tous les types d'utilisateurs, 
en général des non informaticiens. Nous définirons les interfaces du système en 
fonction de ses utilisateurs : acteurs de la télésurveillance, camionneurs, propriétaire.

\end{itemize}
