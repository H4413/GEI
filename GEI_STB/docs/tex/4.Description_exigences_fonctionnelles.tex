\section{Description des exigences fonctionnelles de votre futur système}
\begin{enumerate}

\item  \textbf{Les interfaces pour l'interaction avec le système devront être simples:}\\
Adaptées à chaque type d'utilisateur (propriétaires,techniciens de réservoirs, camionneurs, etc) et technologies(Smartphones,Mobiles).
\begin{itemize}

\item Exigence Non Fonctionnelle: Ergonomie,générecité, Intégration de l'existant
\end{itemize}
\item  \textbf{Le système doit récupérer toutes les données en provenance des capteurs:}\\
La finalité est de sauvegarder les opérations et tous les messages dans le serveur central pendant la production.
\begin{itemize}

\item Exigence Non Fonctionnelle: Traçabilité, Fiabilité, Autonomie
\end{itemize}
\item  \textbf{Le système doit être localisable sur la surface de la planète:}\\
Le système embarqué est muni d'un émetteur GPS permettant sa localisation par le serveur central ou par un client mobile.
\begin{itemize}

\item Exigence Non Fonctionnelle: Fiabilité, Ergonomie
\end{itemize}
\item  \textbf{Toutes les données doivent être à disposition de l'utilisateur de manière simple et portable:}\\
Le système peut communiquer avec les clients de tout type (avec les clients mobiles et avec le serveur central). 
Les informations envoyées sont réduites au nombre de capteurs, types de capteurs, 
seuils tolérés et niveaux actuels détectés. Des commandes spécifiques sont disponibles selon l'information recherchée. 
Il appartient aux différentes plateformes d'envoyer les commandes adéquates en fonction de ce qu'ils recherchent.
\begin{itemize}

\item Exigence Non Fonctionnelle: Traçabilité, Fiabilité, Autonomie, Ergonomie, Généricité
\end{itemize}
\item  \textbf{Le système embarqué doit communiquer avec le serveur central de manière fiable et sans perte:}\\
En cas d'anomalie, le système doit toujours prévenir le serveur central (de façon prioritaire) 
afin de permettre planifier l'intervention.
\begin{itemize}

\item Exigence Non Fonctionnelle: Fiabilité, Autonomie, Maintenabilité
\end{itemize}
\item  \textbf{La configuration des systèmes doit être modifiable à distance sans nécessiter de déplacement vers ces systèmes:}\\
Par exemple faire des modifications de paramètres pour les différents capteurs(seuils), 
ou la mise à jour à des données à travers des commandes depuis le serveur central, ou client mobile.
\begin{itemize}

\item Exigence Non Fonctionnelle: Maintenabilité, Ergonomie, Fiabilité
\end{itemize}
\item  \textbf{Le Système embarqué doit fonctionner de manière autonome sans intervention humaine et cela même en cas de problème environnemental:}\\
Le serveur central doit réceptionner les données provenant des "Capteurs" et "Système" 
des stations en temps réel, dans tous les cas(pannes, dépassement du niveau d'un seuil, problèmes avec un capteur, etc) l
'utilisateur doit être averti pour réaliser le traitement.
\begin{itemize}

\item Exigence Non Fonctionnelle: Autonomie, Robustesse, Fiabilité
\end{itemize}
\item  \textbf{ La consommation électrique doit être minimale pour des raisons évidentes d’autonomie:}\\
La durée de communication entre le système et les utilisateurs doit être 
seulement périodique, sur demande extérieur (par les clients mobiles ou serveur central) ou en cas de problèmes.
\begin{itemize}

\item Exigence Non Fonctionnelle: Autonomie
\end{itemize}
\end{enumerate}

% parmi les exigences non fonctionnelles du paragraphe 3, montrer, pour chaque 
% fonctionnalité, les exigences non fonctionnelles concernant cette 
% fonctionnalité et en étant PLUS PRECIS (Aubry R.)
