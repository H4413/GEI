\section{Étude de faisabilité}

\subsection{Carte embarquée}

\subsection{Énergie}

\subsection{Exigences}
\begin{itemize}

\item Autonomie: Les stations sont souvent situes dans des endroits tr\`es isol\'{e}es et d'acc\`es compliqu\'e. Alors une solution c'est que les ressources naturelles peuvent contribuer  à l'autonomie \'energ\'{e}tique des stations et il est \'{e}vident que la consommation \'{e}lectrique doit \^etre minimale.

\smallskip \item Robustesse et Fiabilit\'{e}: Même en cas de probl\`eme environnemental l'approvisionnement de l'\'{e}nergie \`a la station doit \^etre assur\'{e} sans intervention humaine.

\smallskip \item Protection de l environnement: les stations peuvent se trouver dans des espaces prot\'{e}g\'{e}s donc il faut mettre en place de l'\'{e}nergie non polluant et renouvelable car elle utilise des flux d'\'{e}nergie naturelle(soleil, vent, eau,croissance v\'{e}g\'{e}tale,....).
\end{itemize}

\subsection{Alternatives}
\subsubsection{Energie Solaire Photovolta\"ique}
\begin{enumerate}
\item \textbf{Avantages :}
Les syst\`emes solaires photovoltaïques sont simples et rapides à installer, l'investissement  et rendement sont pr\'{e}visibles à long terme.
Elle r\'{e}pond aux exigences de robustesse et fiabilit\'{e} car est tolérante aux pannes et n\'{e}cessite tr\`es peu de maintenance. 
Elle est exploitable pratiquement partout, la lumi\`ere du soleil \'{e}tant disponible dans le monde entier. 
Aussi, cette production d'\'{e}nergie ne provoque aucun perturbation pour l environnement.

\item \textbf{Risques et Inconv\'enients :}
Ces installations n\'{e}cessitant un apport en soleil, les pays proches des pôles, moins expos\'{e}s aux rayons solaires, sont tr\`es peu rentables.
\end{enumerate}

\subsubsection{Energie \'{e}olienne}
\begin{enumerate}
\item \textbf{Avantages :}
L'\'{e}nergie \'{e}olienne est renouvelable et est id\'{e}ale parce que il s'agit d une forme d'\'{e}nergie ind\'{e}finiment durable.
Elle ne produit pas de d\'{e}chets toxiques ou radioactifs. 
Son temps de fonctionnement est environ 20 ans.

\item \textbf{Risques et Inconv\'enients :}
L'\'{e}nergie \'{e}olienne varie dans le temps.  Elle tourne uniquement s il y a du vent.
La localisation des \'{e}oliennes est d\'{e}pendante de la ressource(le vent) et on ne peut pas les implanter n importe ou. 
Impact sur la faune, les \'{e}tudes ont constat\'{e} que des \'{e}oliennes \'{e}taient responsable de la mort de quelques oiseaux.
\end{enumerate}


\subsubsection{Batterie d Accumulateurs}
\begin{enumerate}
\item \textbf{Avantages :}
Pour une alimentation sans interruption, les batteries stockent l'energie permettant de supplier de quelques minutes à quelques heures.
Il existe les batteries rechargeables qui sont orient\'{e}es pour un fonctionnement avec des panneaux photovoltaïques ou  \'{e}oliennes(dur\'{e}e de vie augment\'{e}, propriét\'{e} anticorrosion grâces à des plaques positives \'{e}paisses, recyclable)

\item \textbf{Inconv\'{e}nients :}
Sont produits lourds donc le frais de port peuvent s av\'{e}rer cons\'{e}quents.
Polluant

\end{enumerate}

\subsection{Choix de Solution}
Apres un analyse des exigences du syst\`eme nous proposons la combinaison de l utilisation des batteries pour le stockage, et pour la production une \'{e}nergie renouvelable pour satisfaire l autonomie \'{e}nerg\'{e}tique.

\subsection{Capteurs}

\subsection{Communication}

Nous avons le choix de plusieurs technologies différentes: la radiodiffusion, GSM, 3G, WiFi, etc.

\subsubsection{Transmetteurs Radio – La radiodiffusion}

La radiodiffusion a la particularité de permettre une communication asymétrique. 
Une station de radio est une installation qui émet des ondes électromagnétiques à l'aide d'un émetteur radio et d'une antenne. 
Un poste de radio ou récepteur radio est un appareil permettant de recevoir les ondes radio, en extraire la modulation et restituer les sons sur un haut-parleur. 

Exemple d'émetteur adapté à notre but au marché: 

SOFREL BOX

\begin{figure}
\begin{center}

% sofrelbox.png

\end{center}
\end{figure}

Les transmetteurs SOFREL BOX sont spécifiquement développés pour le télécontrôle de petites installations isolées et sans énergie.
Très simples à installer et à utiliser, les SOFREL BOX constituent une solution particulièrement adaptée pour les asservissements entre un site isolé et le serveur.
Ils permettent notamment :
L’acquisition d’informations de contrôle (niveaux, compteurs,…)
La communication inter-sites vers un poste local de télégestion SOFREL S500
Doté d’un modem radio intégré pour, le transmetteur déclenche un appel spontané sur le changement vers le S500 de la station.
Hors alarme, les transmissions d’informations se font régulièrement selon une période paramétrable (toutes les 3, 5, 10 ou 15 minutes).

\subsubsection{Transmetteur GSM}

Exemple: CELLBOX

Développé pour la surveillance technique des sites dépourvus de source d’énergie et de toute liaison de communication filaire (RTC, LS/LP…), CELLBOX est un poste local de télégestion autonome communiquant par GSM (900/1800 Mhz).
CELLBOX assure l’acquisition et l’enregistrement d’informations d’alarmes, de comptages et de mesures, et effectue automatiquement différents calculs et bilans.
Etanche, CELLBOX intègre dans son boîtier une pile qui lui procure une autonomie totale de fonctionnement de plusieurs années.
Economique, simple et rapide à mettre en oeuvre, CELLBOX apporte une solution performante pour répondre à de multiples applications : sectorisation de réseaux d’eau, recherche de fuites, télégestion de sites isolés...


\subsubsection{Transmetteur 3G/3G+}

Peu adapté, car le site doit avoir un réseau 3G.

Exemple : VIGICOM VID-1500

\begin{figure}
\begin{center}

% vigicom.png

\end{center}
\end{figure}

VigiCom VID-1500 est un transmetteur  3G spécialement étudié pour équiper tout véhicule ou site isolé. C’est un équipement complet de vidéo-surveillance temps réel par transmission via le réseau cellulaire (3G) réalise une compression vidéo H264 matérielle, transmet depuis deux canaux indépendants et enregistre simultanément sur deux autres canaux séparés les vidéos en qualité DVD sur carte mémoire amovible.

Avantages:
Localisation en temps réel grâce au GPS intégré. 
Grande fiabilité, aucune pièce mécanique, enregistrement sur carte mémoire.
\subsection{Localisation}

\subsubsection{GPS – Global Positioning System}

Le GPS fonctionne grâce au calcul de la distance qui sépare un récepteur GPS et plusieurs satellites. Les informations nécessaires au calcul de la position des 31 satellites étant transmise régulièrement au récepteur, celui-ci peut, grâce à la connaissance de la distance qui le sépare des satellites, connaître ses coordonnées.

Nous pouvons aisement utiliser un module GPS embarqué pour savoir à tout momment les positions exacte des sites.

Quelques modules de GPS embarqué:

Exemple 1 : Module SKM53 avec antenna

\begin{figure}
\begin{center}

% modulegps.png

\end{center}
\end{figure}

La série SkyNav SKM53 avec antenne permet une navigation de haute performance dans les applications les plus rigoureuses.
Il est basé sur les caractéristiques de haute performance de la MediaTek 3327/3329 architecture mono-puce, sa sensibilité de suivi 165dBm étend la couverture de positionnement. Il est la solution la plus simple et commode pour être embarqué dans un appareil portable.

Prix: 32 euros

Exemple 2: Embedded GPS/GALILEO PCI Express Mini Card
Cette carte peut être relié directement à l'ordinateur par une entré PCI.
Prix: environ 50 euros.

\subsection{Serveur}
\vfill
\pagebreak
