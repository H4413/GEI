% Dossier de faisabilité - H4413~-~GEI

\documentclass[twoside]{article}
\usepackage{hyperref}


\usepackage{graphicx}
\usepackage{subfig}
\usepackage{placeins}


% Unicode encoding  
\usepackage[utf8]{inputenc}


% Colorfull Text
\usepackage{xcolor}


% Language settings:
\usepackage[french]{babel}

\usepackage[T1]{fontenc}


% Tables
\usepackage{array}
\usepackage{longtable}


% Hyperrefferences  
\usepackage{hyperref}


% Font settings:
\usepackage{utopia}


\title{Dossier de faisabilité}
\author{H4413~-~GEI}
% Page layout settings
\usepackage{geometry}
\geometry{
	a4paper,  % 21 x 29,7 cm
	body={160mm,240mm},
	left=30mm, 
	top=25mm,
	headheight=7mm, 
	headsep=4mm,
	marginparsep=4mm,
	marginparwidth=27mm
}


% Spacing:
\usepackage{setspace}


% Headers and footers:
\usepackage{fancyhdr}
\pagestyle{fancy}
          \fancyhf{}
          \fancyfoot[LE,RO]{\textcolor[gray]{0.3}{\thepage}}
          % Rulers width
          \renewcommand{\footrulewidth}{.3pt}
          \renewcommand{\headrulewidth}{.0pt}
\fancyfoot[LO,RE]{\textcolor[gray]{0.3}{H4413~-~GEI}}
\fancyfoot[CO,CE]{\textcolor[gray]{0.3}{Dossier de faisabilité}}


% Vars & functs
\newcommand\PIXPATH{./docs/pics}
\newcommand\SRCPATH{./docs/src}
\newcommand\Object{Étude sur la faisabilité du projet.}
\newcommand\Version{1}
\renewcommand{\labelitemi}{$\diamond$}
\renewcommand{\labelenumi}{(\alph{enumi})}


% Begining of the document
\begin{document}

	%Including all the files:

    % Fichier ./docs/tex/00.a.premiere_page.tex

% Front Page 

% Title:
\maketitle

\thispagestyle{empty}

\hfill\\
\vfill

% Objet
\section*{Objet}
\Object

% Version actuelle
\section*{Version}
\Version

% Picture
\begin{center}
    \includegraphics[width=3cm]{\PIXPATH/insa}\hfill\\
%    \includegraphics[width=5cm]{\PIXPATH/frontPage}
\end{center}

\pagebreak

    % Fichier ./docs/tex/00.b.suivi.tex

% Suivi du document

% Modifications
\section*{Modifications du document}

\begin{center}
\begin{longtable}{|m{14mm}||m{36mm}|m{36mm}|m{60mm}|}
\hline
Version & Auteur & Date & Modification\endhead \hline
% Version
1
& % Auteur
H4413~-~GEI
& % Date
20 décembre 2010
& % Modification
Création
\\\hline
% Version

& % Auteur

& % Date

& % Modification

\\\hline
% Version

& % Auteur

& % Date

& % Modification

\\\hline
% Version

& % Auteur

& % Date

& % Modification

\\\hline
\end{longtable}
\end{center}

% Validations

\section*{Vérifications et validations du document}

\begin{center}
\begin{longtable}{|m{15mm}|m{36mm}|m{36mm}|m{60mm}|}
\hline
 & Responsable & Date & Remarques\endhead \hline
% Validé/vérifié par

& % Responsable

& % Date

& % Remarques

\\\hline
% Validé/vérifié par

& % Responsable

& % Date

& % Remarques

\\\hline
% Validé/vérifié par

& % Responsable

& % Date

& % Remarques

\\\hline
% Validé/vérifié par

& % Responsable

& % Date

& % Remarques

\\\hline
\end{longtable}
\end{center}

\pagebreak

    % Fichier ./docs/tex/01.Introduction.tex

\section{Introduction}

    % Fichier ./docs/tex/02.Documents_applicables.tex

\section{Documents applicables et de référence}

Le Dossier de Gestion et de Structuration de la Documentation s'applique à ce document.
    % Fichier ./docs/tex/03.Analyse_de_l_existant.tex

\section{Analyse de l'existant}

    % Fichier ./docs/tex/04.Etude_de_faisabilité.tex

\section{Étude de faisabilité}

\subsection{Carte embarquée}

\subsection{Énergie}

\subsection{Capteurs}

\subsection{Communication}

\subsection{Localisation}

\subsection{Serveur}

    % Fichier ./docs/tex/05.Conclusions.tex

\section{Conclusions}

% The end
\end{document}

