\section{Introduction}

Le Comité pour la Protection de l'Environnement de l'UE (COPEVUE) est en charge de surveiller un vaste nombre de site en Europe. Ces sites surveillés peuvent être très différents, des zones méditerranéennes subissant de nombreux incendies pendant les étés aux régions nordiques difficiles d'accès pendant l'hiver. 

Ces sites nécessitent une action régulière de l'homme pour des actions diverses comme du pompage, de l'entretien où encore des études sur la faune et la flore. Pour permettre une surveillance des sites plus efficace des sites, le COPEVUE souhaite mettre en place un suivi à distance (monitoring) pour leurs sites pour : 

\begin{itemize}
\item Surveiller en temps réel l'ensemble des sites et réduire les risques environnementaux
\item Centraliser les informations pour un meilleur suivi
\item Optimiser les actions effectuées pour les sites pour réduire les coûts
\end{itemize}

Les domaines d'exigences se situent à plusieurs niveaux : 

\begin{itemize}
\item l'autonomie du système embarqué
\item les communications du système embarqué avec le système central
\item la surveillance des environnements avec les capteurs et les interactions locales
\item la qualité des interfaces utilisateurs
\end{itemize}

Ce document a pour but d'étudier la faisabilité de ce système de monitoring par rapport aux principaux domaines d'exigences formulées par le comité. Nous traiterons ces domaines uns par uns pour certifier les possibilités de réalisation de votre système.

%NOTE DE RELECTURE : présenter l'ordre de traitement des domaines que va
%couvrir le dossier (capteurs, énergie, etc.)
\vfill
\pagebreak
