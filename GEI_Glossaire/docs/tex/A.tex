\section{A}

%Quentin
\begin{description}

\item[Axe d'amélioration] Un axe d'amélioration est un grand domaine qui doit 
progresser au sein d'un système. Un axe est déduit de l'analyse de
 l'existant. Ils sont ensuite répartis en trois catégories : 

\begin{itemize}
\item[Principaux] : Ces axes permettront d'avoir un impact fort et direct sur le système. Ils sont à traiter en priorité.
\item[Marginaux] : Ces axes feront progresser mais n'auront peut être pas un impact fort. Ils sont à traiter en suivant.
\item[Faux] : Ces axes ne permettront pas d'augmenter la valeur du système même si ils pourraient être attractif.
\end{itemize} 

\item[API] 
Est une interface fournie par un programme informatique. Elle permet 
l'interaction des programmes les uns avec les autres, de manière analogue à 
une interface homme-machine.Du point de vue technique une API est un ensemble 
de fonctions mises à disposition par une bibliothèque logicielle, un système 
d'exploitation ou un service.


\end{description}
