\section{E}

\begin{description}

\item[Exigence Non-Fonctionnelle] Ce type d'exigence spécifie quelque chose que 
le système livré doit avoir

\item[ERP] est un progiciel de gestion des ressources d'une Entreprise. 
Il permet d'avoir un portail centralisé regroupant l'ensemble des interventions 
des entreprises et l'historique des valeurs des capteurs.

\item[Emetteur/Récepteur radio]
Dans les réseaux informatiques du type Ethernet le émetteur/récepteur est 
intercalé entre le câble qui forme le réseau (paire torsadée ou coaxial) et 
l'interface physique sur la machine. Il permet donc le rattachement de la 
station au réseau.

\item[Exigence Fonctionnelle]
Elles constituent le sujet principal et la matière fondamentale pour la création 
du système. Elles sont mesurées par des moyens concrets comme les valeurs de données, 
la logique des processus décisionnels ou les algorithmes.

\end{description}
