\section{Introduction}

Ce dossier a pour objectif la description du déroulement du projet
"Solution de monitoring à distance de sites isolés" commandé par le
COPEVUE et réalisé par la Team H4413. Il regroupe tous les éléments
indispensables au projet et constitue une référence pour en surveiller le
bon déroulement.\\

Le but du PMP est de :
\begin{itemize}
\item Guider le développement pour le mener à bien dans les délais
\item Permettre la valorisation, le suivi et le contrôle des indicateurs
quantitatifs
\end{itemize}
\hfill\\

Le PMP couvre toutes les phases du projet :
\begin{itemize}
\item Spécification préalable
\item Réalisation
\item Validation
\end{itemize}


\subsection{Documents de référence}

Le PMP a pour référence les documents suivants :
\begin{description}
\item[Cours de RA] TODO : cibler le cours à citer
\item[Dossier de réponse à l'appel d'offre] réalisé par la Team H4413 et
accepté par le COPEVUE.
\end{description}


\subsection{Documents applicables}

Heu, c'est quoi ?


\subsection{Terminologie}

On détaille ici les termes techniques et acronymes utilisés dans ce
dossier.

\begin{description}
\item[COPEVUE] : COmité pour la Protection de l'EnVironnement de l'Union
Européenne.
\item[PMP] : Plan de Management de Projet. Présent document.
\item[STB] : Spécification Technique des Besoins ; dossier réalisé lors de
l'étude préalable afin de répondre à l'appel d'offre et spécifiant les
besoins du futur système en termes d'exigences fonctionnelles et
non-fonctionnelles.
\end{description}
