\section{Présentation du projet}

\subsection{Rappel du problème}

Le Comité pour la Protection de l'EnVironnement de l'UE (COPEVUE) est en
charge de surveiller un vaste nombre de site en Europe. Ces sites
surveillés peuvent être très différents, des zones méditerranéennes
subissant de nombreux incendies pendant les étés aux régions nordiques
difficiles d'accès pendant l'hiver. 

Ces sites nécessitent une action régulière de l'homme pour des actions
diverses comme du pompage, de l'entretien où encore des études sur la faune
et la flore. Pour permettre une surveillance des sites plus efficace des
sites, le COPEVUE souhaite mettre en place un suivi à distance (monitoring)
pour leurs sites pour : 

\begin{itemize}
\item Surveiller en temps réel l'ensemble des sites et réduire les risques environnementaux
\item Centraliser les informations pour un meilleur suivi
\item Optimiser les actions effectuées pour les sites pour réduire les coûts
\end{itemize}

Les domaines d'exigences se situent à plusieurs niveaux : 

\begin{itemize}
\item l'autonomie du système embarqué
\item les communications du système embarqué avec le système central
\item la surveillance des environnements avec les capteurs et les interactions locales
\item la qualité des interfaces utilisateurs
\end{itemize}


\subsection{Références contractuelles}

Pour le moment, les références contractuelles connues sont les suivantes :

\begin{itemize}
\item Appel d'offre émis par la COPEVUE
\item Dossier de réponse à l'appel d'offre proposé par la Team H4413 et
accepté par la COPEVUE ; ce dossier est principalement composé de :
    \begin{itemize}
    \item Étude de faisabilité
    \item Spécification technique des besoins
    \item Ébauche de conception détaillée
    \item Plan d'Assurance Qualité Projet
    \end{itemize}
\end{itemize}


