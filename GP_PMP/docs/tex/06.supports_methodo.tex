\section{Supports méthodologiques et moyens à mettre en oeuvre}

\subsection{Moyens matériels}

L'équipement basique d'une équipe comprend les éléments suivants : 

\begin{itemize}
\item Un local par équipe
\item Un ordinateur de travail personnel portable par membre de l'équipe
\item Une station d'accueil complète par membre de l'équipe :
    \begin{itemize}
    \item Un couple clavier/souris
    \item Un deuxième écran
    \end{itemize}
\item L'accès aux ressources communes : imprimante, serveur, machine à café
\end{itemize}

\vskip 6pt

Chaque équipe dispose de plus d'un budget suffisant pour acquérir les matériels
nécessaires à l'accomplissement de ses tâches, tels que kit de
développement pour l'équipe de Génie Électronique.

\subsection{Moyens logiciels}

La solution mise en place privilégiant les logiciels libres, les moyens
utilisés par les équipes seront en accord avec cette philosophie. La
distribution standard sera Ubuntu, mais chaque collaborateur sera libre de
la remplacer par celle de son choix si il l'estime nécessaire et que cela
ne nuit pas à sa productivité.\\

Au niveau méta, nous acquérirons toutefois deux choses :
\begin{itemize}
\item Un compte github pro pour toute la durée du projet
\item Un compte Google pro afin de disposer pleinement des services
hébergés par Google (gmail, gdoc, etc.)
\end{itemize}

\vskip 6pt

Enfin, une instance RedMine sera installée sur le serveur de l'entreprise
afin de suivre l'évolution du projet.


\subsection{Moyens humains}

La composition de l'équipe prenant en charge le projet est décrite dans la
section 3 du PMP.

\subsection{Évaluation de la charge}

La charge de travail est évaluée par le chef de projet selon les méthodes
en vigueur dans l'entreprise  et soumise à l'approbation des experts et du
responsable qualité.\\

Chaque collaborateur devra reporter son temps de travail avec l'outil
RedMine.

\subsection{Réalisation du bilan humain}

Un bilan quotidien est réalisé de manière informelle au cours de la pause
café par le CdP avec les collaborateurs qu'il a alors l'occasion de
rencontrer.\\

Un bilan hebdomadaire de suivi de projet est prévu. Cette réunion regroupe
le CdP, le RQ, et les experts assistés des membres de leurs équipes ayant
des remarques à formuler. Ces bilans provisoires seront l'occasion de faire
le point sur l'avancement du projet, faire remonter les problèmes et les
confronter aux points de vue des différents intervenants.\\

Un bilan mensuel (un des bilans hebdomadaires) regroupera l'ensemble des
membres de l'équipe et sera suivi d'une activité de \textit{team-building},
par exemple un repas dans un restaurant.

\subsection{Moyens financiers}

Présent dans la version finale du PMP
