\section{Méthodes/outils/techniques/règles/normes}

\subsection{Outils}
Utilisation d'une fiche de suivi d'avancement identique à celle de la 1ère 
phase du projet.
La documentation est au format Latex.
Le développement se fait sous VIM et est versionné avec Git.
Cf PAQP pour une description des outils.

\subsection{Règles et normes}
Les règles de développement sont celles définies par les best practices et 
normes du langage utilisé pour le sous-projet.
Le langage doit être unique pour le sous-projet.

Le respect des API définies pour le projet est le premier critère à prendre en 
compte.

Pour un soucis de modularité, tout le code doit être découpé en modules pour
chaque sous-tâche que doit réaliser le logiciel.

Pour la maintenabilité :

Toute fonction et toute classe créée doit être décrite suffisament dans le code 
via la doc disponible dans le langage ou en commentaire si le langage ne le
permet pas. 
Tout algorithme doit être décrit et sa fonction doit être explicitement donnée.

Portabilité :

Le code produit doit être adapté à toutes les plateformes dans la mesure du 
possible. Si ce n'est pas possible, le code devra être facilement modifiable 
pour être adapté à une plateforme quelconque.

Fiabilité :

Enfin tout le code doit être testé. L'utilisation d'outils de 
couverture de tests permettrait d'être sûr. 
Des tests d'intégration doivent aussi être présents.

\subsection{Autres points à aborder}


