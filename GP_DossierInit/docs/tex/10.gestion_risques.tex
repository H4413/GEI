\section{Gestion des risques}

Divers risques peuvent entraîner des retards dans l'exécution du projet et
doivent être pris en compte dans sa conduite. Certains risques sont liés au
projet tandis que d'autres proviennent de l'équipe menant à bien le projet.

\subsection{Risques liés à l'équipe}

Les risques liés à l'équipe sont causés par ses membres. Les motifs de
retards peuvent être les suivants :

\begin{itemize}

\item Epidémie majeure de grippe
\item Mauvaise qualité de service du réseau Insa rendant inaccessible tout
ou partie d'Internet (et plus particulièrement le tableau de bord Google
Spreadsheet ainsi que le dépôt git hébergé sur \url{www.github.com})
\item Membre insubordonné et dispersé en réunion
\item GEI peu efficace durant les réunion et très peu efficace en dehors
des réunion
\item Chef de Projet inapte à motiver le GEI
\item Temps d'adaptation aux technologies employées au sein de l'hexanôme
(rapport en \LaTeX, versionnement par Git)
\item Compétences techniques du GEI insuffisantes
\item Perfectionnisme

\end{itemize}


\subsection{Risques liés au projet}

Les risques liés aux projets sont les suivants :

\begin{itemize}

\item Cahier des charges mouvant ; l'appel d'offre étant ferme et émis, ce
risque peut être écarté.
\item Solution trop compliqué donc plus longue à mettre en oeuvre

\end{itemize}

\vfill
\pagebreak
