\section{Gestion des risques}

Divers risques peuvent entraîner des retards dans l'exécution du projet et
doivent être pris en compte dans sa conduite. Certains risques sont liés au
projet tandis que d'autres proviennent de l'équipe menant à bien le projet.

\subsection{Risques liés à l'équipe}

Les risques liés à l'équipe sont causés par ses membres. Les motifs de
retards peuvent être les suivants :

\subsubsection{Non maîtrise des outils}

Les technologies mises en oeuvres pour répondre à l'appel d'offre de la
COPEVUE ne sont pas maîtrisées par l'ensemble de l'équipe :
\begin{itemize}
\item Le système de versionnement, Git, n'a jamais été utilisé par trois
membres de l'équipe
\item L'outil de rédaction, \LaTeX, n'est pas familier à deux des membres
de l'équipe.
\end{itemize}
\hfill\\
Le risque est \textbf{probable}.\\

Afin d'agir contre ce risque, les membres maîtrisant pleinement les
technologies employées auront sans doute une charge de travail légèrement
supérieure au reste de l'équipe au début du projet, afin de former leurs
collaborateurs débutants.


\subsubsection{Dérive par rapport à l'objectif}

L'équipe n'ayant encore jamais répondu à un appel d'offre ni conçu de
système embarqué devant répondre au cahier des charges de la COPEVUE, elle
risque de ne pas savoir dans direction aller. Le risque est
\textbf{probable} et peut prendre plusieurs formes :
\begin{itemize}
\item \textbf{Surqualité} des livrables par rapport au standard exigé
(perfectionnisme)
\item \textbf{Dispersion} de l'équipe dans de mauvaises directions
\end{itemize}
\hfill\\

Toutefois, l'importance de la documentation fournie ainsi qu'une gestion de
projet rigoureuse devrait permettre de minimiser ce risque. Des réunions de
début et de fin de séance semblent être un minimum.


\subsubsection{Autres risques identifiés}

En plus des deux principaux risques évoqués ci-dessus, on peut noter les
risques suivant, qui bien que moins probables, doivent être mentionnées et
être pris en compte.

\begin{itemize}

\item Epidémie majeure de grippe
\item Mauvaise qualité de service du réseau Insa rendant inaccessible tout
ou partie d'Internet (et plus particulièrement le tableau de bord Google
Spreadsheet ainsi que le dépôt git hébergé sur \url{www.github.com})
\item Membre insubordonné et dispersé en réunion

\end{itemize}

\subsection{Risques liés au projet}

Les risques liés aux projets sont les suivants :

\begin{itemize}

\item Cahier des charges mouvant ; l'appel d'offre étant ferme et émis, ce
risque peut être écarté.
\item Solution trop compliqué donc plus longue à mettre en oeuvre

\end{itemize}

\vfill
\pagebreak
