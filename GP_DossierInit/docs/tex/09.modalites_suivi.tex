\section{Modalit� de suivi}

\subsection{Les r�gles de suivi}

On �tablit les r�gles de suivi suivantes :

\begin{itemize}
\item Les t�ches sont distribu�es par mail par le Chef de Projet
\item Un rappel de l'ensemble des t�ches assign�es aux diff�rents membres
de l'�quipe est disponible pour tous sur le tableau de bord de l'�quipe
(google Spreadsheet)
\item Chacun est tenu de mettre � jour les t�ches le concernant.
\item Le CdP valide les drafts associ�s aux t�ches sur le fond
\item Le RQ valide les livrables interm�diaires associ�s aux t�ches sur la
forme
\end{itemize}

Par ailleurs l'ensemble du travail de l'�quipe est versionn� en utilisant
git associ� au service github.com, ce qui permet de le mettre � disposition de tous
via Internet, de se pr�munir efficacement contre les effacements
malencontreux de dossiers et de versionner facilement drafts, livrables
interm�diaires et livrables.

\subsection{Les outils utilis�s}

Le travail de l'�quipe est versionn� au moyen de \textit{Git} et h�berg�
sur la plateforme \url{www.github.com}.\\
Par ailleurs, un "tableau de bord" du projet a �t� mis en place en
utilisant Google Spreadsheet. Il permet de suivre l'assignation des t�ches
et leur avancement.\\
Enfin, le planning de r�alis� et tenu � jour au moyen du logiciel \textit{GanttProject}. 


\subsection{Les proc�dures de r�vision du planning}

� chaque s�ance, le planning est mis � jour par le CdP.
