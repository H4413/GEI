\section{Organisation du travail}

Dans cette partie on d�taillera les r�les des diff�rents membres de l'�quipe
menant � bien le projet.


\subsection{Chef de Projet (Cl�ment Geiger)}

Coordinateur du projet et r�f�rent aupr�s des clients.


\subsection{Responsable Qualit� (Hugo Pastore de Cristofaro)}

Responsable de la qualit� des livrables rendus aux clients et garant du respect
des proc�dures de r�daction et Bonnes Pratiques qu'il aura mis
en place avant le d�but du projet.

\subsection{GEI}

Le Groupe d'�tude Informatique est compos� des personnes suivantes :

\begin{itemize}
\item Karen Abanto
\item Victor Borges
\item Rapha�l Liz�
\item Quentin Villers
\end{itemize}

\hfill\\

Le GEI est le groupe d'experts techniques menant les diff�rentes �tudes
n�cessaires pour apporter une r�ponse � l'appel d'offre de la COPEVUE.


\subsection{Workflow}

Le CdP a mis en place un outil de distribution des t�ches et de suivi du
travail utilisant Google SpreadSheet. Cet outil permet de d�tailler les
t�ches n�cessaire � la r�alisation de chaque livrable, les assigner � une
personne et suivre leur avancement en pourcentage. Chaque t�che comporte
une sous-t�che "Draft" et une autre "Livrable".\\
Une fois un draft termin�, il est valid� sur le fond par le CdP, et la
personne ayant �t� assign� � la t�che peut passer � la mise en forme
(partie "Livrable" de la t�che). Cette derni�re partie, une fois termin�e,
est valid�e sur la forme par le RQ.
