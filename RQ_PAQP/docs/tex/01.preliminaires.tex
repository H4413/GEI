% Préliminaires

\section{Préliminaires}
    \subsection{Cadre du PAQP}
    Le PAQP suivant est à utiliser pour le projet GEI dans la réponse à l'appel 
d'offre du COPEVUE pou un système de monitoring de sites isolés.
Il sera appliqué pour l'ensemble du projet.
    \subsection{Logiciels concernés par le PAQP}
    Le PAQP concerne tous les logiciels produits par H4413 dans le cadre du
projet.
    \subsection{Responsabilités associées au PAQP}
    Tous les membres de H4413 ainsi que les sous-traitants sont tenus de
respecter ce PAQP.
    \subsection{Procédure d'évolution du PAQP}
    Le PAQP ne peut être modifié sans l'accord du Responsable Qualité projet.
    Pour toute demande de modification, s'adresser à son responsable qualité qui 
contactera le Responsable Qualité Projet.
    
    Si une modification est autorisée. C'est un Responsable Qualité qui s'en charge.
    Cette modification est clairement décrite dans le fichier de suivi des 
modifications et ne prendra effet qu'après revalidation par le Responsable 
Qualité Projet qui consignera cette validation dans le fichier de suivi.
    \subsection{Procédure à  suivre en cas de non-application du PAQP}
    \subsubsection{Non application}

    En cas de non application du PAQP, le logiciel, livrable ou tout autre forme 
de travail soumis au PAQP ne sera pas validé par le Responsable Qualité.
    Le Responsable Qualité indique cette non validation sur le tableau de bord 
en précisant les points non respectés.
    Le coupable, qui consulte régulièrement le tableau de bord et qui suit la 
validation de ses travaux, lit la remarque et fait la modification adéquate.
    En cas de non correction et à l'approche de la date limite précisée par le 
chef de projet ou sous-projet. Une intervention orale sera faite envers le 
coupable.

    \subsubsection{Demande de dérogation}

    Dans un cas justifiable, une dérogation au PAQP peut être donnée. Pour cela,
il faut envoyer une Demande de dérogation au Responsable Qualité.
    Suivant l'acceptation, des mesures spécifiques seront prises après 
concertation avec le demandeur.
