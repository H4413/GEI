\section{Gestion de configuration}

\subsection{Conventions d'identification}
Les livrables possèdent des tableaux, en début de document, permettant
de tracer son historique: qui a modifié le document, quand, avec un
commentaire; qui a relu/validé le livrable, quand, avec un commentaire.

\subsubsection{Responsabilités}

Chacun est responsable de ses entrées dans les tableaux de suivi et sur le 
tableau de bord.

Le responsable qualité vérifie ces entrées.

Le responsable qualité est responsable 

\subsubsection{Procédures de gestion de la configuration}

Les démarches de développement utilisent le cycle en V.

Les différentes tâches sont données et suivies sur le tableau de bord. Cependant, 
le reponsable de séance rappelle en début de séance quelles sont les tâches de 
chacun pour la séance courante.

\subsection{Gestion des ressources partagées}

Les fichiers de chaque projet doivent être nommés avec le nom du projet puis du 
sous_projet. (Ex : GEI_Alimentation_main.cpp)
Toutes les ressources sont déposées sur le dépôt de versionnement. Le travail 
collaboratif est géré par ce dépôt de versionnement.

