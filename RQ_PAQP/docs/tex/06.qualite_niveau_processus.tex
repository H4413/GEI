% Qualité au niveau du processus

\section{Qualité au niveau du processus}

    \subsection{Présentation de la démarche de développement au niveau Projet}

        \subsubsection{Généralités}
        
        \subsubsection{Phase d’étude préalable}

        \subsubsection{Phase d’étude détaillée}

        \subsubsection{Phase d’intégration système}

        \subsubsection{Phase de validation système}

        \subsubsection{Phase de mise en \oe{}uvre sur site pilote}

    \subsection{Règles de qualité pour l’ingénierie concurrente}

        \subsubsection{Règles sur la rédaction d’un cahier des charges d’un 
            sous-projet}

            Le cahier des charges doit respecter le plan-type donné par le
            Responsable Qualité. Vous pouvez trouver une Best Practice pour la
            rédaction d'un cahier des charges ainsi que le plan type dans le 
            répertoire GEI_BP.

        \subsubsection{Règles sur la définition précise des résultats attendus 
            pour chaque sous-projets}

            Chaque sous-projet doit remettre les documents produits au fil de
            l'eau. A la fin du sous-projet, la cohérence avec le cahier des 
            charges doit être évaluée et remise au Responsable Qualité.            

        \subsubsection{Règles sur le suivi qualité des sous-projets}

            Les sous-projets sont suivis au fur et à mesure de l'avancement et 
            ne se limitent pas à une validation finale.

        \subsubsection{Règles sur la définition de critères d’acceptation des 
            sous-projets avant intégration}

            Un sous-projet est validé s'il respecte les dispositions qualité 
            imposées, qu'il respecte les exigences du cahier des charges(lui-
            même validé) et que les API correspondent à celles définies.

    \subsection{Présentation des démarches de développement au niveau 
            sous-projets}

        \subsubsection{Liste des processus de développement susceptibles d’être 
            retenus pour le développement des sous-projets}

            Le processus de développement du cycle en V sera utilisé pour le
            développement des sous-projets.

        \subsubsection{Description du cycle de développement n\degre 1}

            %\par{Liste des étapes}

            %\par{Étape numéro 1}

            \begin{description}
                \item[Documents en entrée]
                \item[Documents en sortie]
                \item[Conditions de validation de l'étape]
            \end{description}

            %\par{Suivi de projet}
\pagebreak
