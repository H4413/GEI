\section{Formalisation des documents produits}
\subsection{Processus de création d'un nouveau document}
Chaque nouveau document à l'exception des drafts est créé par le responsable 
qualité ou le chef de projet. Il s'agira toujours d'un fichier au format Latex.
\subsection{Présentation des documents}
\subsubsection{Draft}
Les drafts sont placés dans un dossier spécifique et ne sont soumis à aucune 
règle structurelle. Il est par ailleurs conseillé de respecter la règles des 
5 lignes : une idée doit pouvoir être transmise en 5 lignes.
Les documents de type draft doivent contenir l'auteur, la date de rédaction, 
le positionnement dans le projet, la philosophie du draft, l'objectif et les 
idées directrices.
\subsubsection{Livrables intermédiaires, livrables finaux}
Les livrables intermédiaires et finaux devront respecter les plans type donnés 
en annexe. 
Tous les livrables contiennent des informations de suivi qui doivent êtres 
mises à jour à chaque modification par un rédacteur. La validation et 
vérification sont aussi contenues dans un fichier de suivi 00.suivi.tex.
