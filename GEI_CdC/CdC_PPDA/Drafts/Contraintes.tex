
\section{Contraintes}
\subsection{Exigences non Fonctionnelles}	
\begin{description}		
\item [Intégration de l'existant]\hfill\\
L'application doit permettre aux différents utilisateurs de récupérer des rapports, des messages clairs,et de transmettre en tout moment et de manière fiable des demandes d'intervention pour la maintenance.\\

\item [Ergonomie]\hfill\\
Les interfaces de contrôle du système seront intégralement graphiques, puisque ce sont des non informaticiens qui vont devoir utiliser le système.\\

\item [Traçabilité]\hfill\\
 L'application doit pouvoir transmettre tous les messages reçus à l'archivage dans les bases de données.\\

\item [Évolutivité]\hfill\\
L'application doit être configurable pour ameliorer la performance et fonctionnalites(Le nombre de capteurs raccordés à une station, Le nombre de stations dans un réseau).

\item [Portabilité]\hfill\\
L'application devra fonctionner sur des environnements et plate-formes différents de smartphone (iOS, Android, WebOs, MeeGo).

\end{description}

	
\subsection{Complexité}
	
L'application pour le système ne présente pas de grandes difficultés, le but est de veiller à ce que l'IHM soit adaptée aux utilisateurs.

La vraie difficulté viendra de l'intégration au sein de la même interface des modules de gestion de la maintenance et d'exploitation statistique des données.

