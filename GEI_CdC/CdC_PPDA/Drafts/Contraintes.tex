%!TEX encoding = UTF-8 Unicode
\documentclass [a4paper] {report}
\usepackage[utf8]{inputenc}
\usepackage[francais]{babel}
\usepackage[top=2cm, bottom=2cm, left=2cm, right=2cm]{geometry}
\usepackage{hyperref}
\usepackage{eurosym}
\usepackage{graphicx}
\usepackage{longtable}
\begin {document}
\section{Contraintes}
	
\begin{description}		
\item [Intégration de l'existant]\hfill\\
L'application doit permettre aux différents utilisateurs de récupérer des rapports, des messages clairs,et de transmettre en tout moment et de manière fiable des demandes d'intervention pour la maintenance.\\

\item [Ergonomie]\hfill\\
Les interfaces de contrôle du système seront intégralement graphiques, puisque ce sont des non informaticiens qui vont devoir utiliser le système.\\

\item [Traçabilité]\hfill\\
 L'application doit pouvoir transmettre tous les messages reçus à l'archivage dans les bases de données.\\

\item [Évolutivité]\hfill\\
L'application doit être configurable pour ameliorer la performance et fonctionnalites(Le nombre de capteurs raccordés à une station, Le nombre de stations dans un réseau).

\end{description}

	
\subsection{Complexité}
	
L'application pour le système ne présente pas de grandes difficultés, le but est de veiller à ce que l'IHM soit adaptée aux utilisateurs.

La vraie difficulté viendra de l'intégration au sein de la même interface des modules de gestion de la maintenance et d'exploitation statistique des données.

\end{document}