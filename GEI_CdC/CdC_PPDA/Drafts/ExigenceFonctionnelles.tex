
\section{Exigences fonctionnelles}

\subsection{Fonctions de base}

\begin{description}
	\item[ Monitoring à distance]\hfill\\
		L'interface devra pouvoir permettre de visualiser l'ensemble des stations. On devra afficher une carte localisant toutes les stations et mettre en évidence les régions dans lesquelles elles se situent. Il devra être possible de localiser rapidement un problème ou de sélectionner les différentes entités pour voir leur état.\\
		
	\item[Connexion locale à une station]\hfill\\
	 	Lorsque l'utilisateur est dans la station pour faire une maintenance, il peut faire des demandes de consultation au système embarquée, des modifications sur les paramètres de la station avec son PDA en se connectant au serveur web de la station via bluetooth. 
		
	\item [Réception des messages depuis la station]\hfill\\
		L'interface devra signaler lorsqu'un station envoie un message. \\

	\item [Envoi des messages vers la station]\hfill\\
		L'application devra pouvoir sélectionner une station pour lui demander d'envoyer son état ou la valeur de ses capteurs. Elle devra également pouvoir envoyer des mises à jour via le fichier de configuration.\\

	\item [Accès à la base de donnes]\hfill\\
		L'application devra accéder à la base de données afin de pouvoir visualiser l'ensemble des messages ou rapports associés à une station. L'interface devra être simple et on ne doit pas pouvoir modifier ces données pour garantir leur fiabilité.
\end{description}


\subsection{Contraintes d'utilisation}

\begin{description}
	\item [Sécurité]\hfill\\
		L'ensemble des échanges devra être sécurisé pour éviter qu'une personne extérieure puisse accéder aux données ou se servir du réseau mis en jeu à des fins personnels\\

	\item [Contrôle d'accès]\hfill\\
		 Il existe différents profils d'utilisateurs.Après l'authentification, chaque utilisateur aura sur un ensemble unique droits d'accès aux informations stockes dans la basse de données nécessaire pour mettre en place la politique de sécurité.
 	
	\item [Ergonomie]\hfill\\
		L'interface devra être simple et concise pour s'adapter à des utilisateurs à faibles connaissances informatiques.\\
\end{description}	
	
