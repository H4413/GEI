% Configuration cible

\section{Configuration cible}

    \subsection{Matériel et logiciels}
        La configuration du matériel se fera une première fois
        par le technicien qui installera le matériel, puis via le
        réseau, ou, si nécessaire, par un technicien sur place.
        
        Le système embarquera un serveur web permettant la configuration
        sur place (sur du Bluetooth par exemple). La configuration à 
        distance se fera par le réseau GPRS. 

    \subsection{Stabilité}
        Les transmission par GPRS sont le point faible de la
        communication avec le serveur central. En effet, selon la
        localisation géographique de la station, la réception (et
        l'émission) peuvent être entravées par des obstacles naturels,
        ou par une faible couverture réseau. On peut envisager
        l'utilisation d'une antenne, voir d'un amplificateur (au
        détriment de l'économie d'énergie).

        Le système doit être fait de telle manière qu'il nécéssite le
        moins d'intervention sur site, notament en cas de plantage:
        il doit être capable de se relancer et d'être à nouveau
        opérationnel dans un délai raisonnable.
