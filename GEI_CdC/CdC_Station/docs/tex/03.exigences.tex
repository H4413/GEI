% Exigences fonctionnelles

\section{Exigences fonctionnelles}

    \begin{enumerate}

    % Mise en forme liste:
  %  \renewcommand{\labelenumi}
  %          {FO n\degres \arabic{enumi}~:}
  %  \renewcommand{\labelenumii}
  %          %{\labelenumi.\arabic{enumii}~:}
  %          {FO n\degres \arabic{enumi}.\arabic{enumii}~:}
    
    % Texte:
        \item Monitoring à distance
            \begin{enumerate}
                \item Consultation de l'état de charge des capteurs;
                \item Consultation des valeurs d'un ou plusieurs
                         capteurs.
                \item Émission d'erreur en cas de problème matériel
                            ou de seuil (cuve);
            \end{enumerate}

        \item Maintenance à distance
        
        \item Maintenance sur site

        \item Traitements sur station
            \begin{enumerate}
                \item Interrogation des capteurs;
                \item Traitement des données capteur;
                \item Traitement des données GPRS;
                \item Traitement des données GPS;
                \item Traitement des données Bluetooth;
                \item Serveur web de configuration.
            \end{enumerate}
    \end{enumerate}

\section{Exigences non fonctionnelles}
    \begin{description}
        \item[Robustesse:]\hfill\\
            le système doit toujours revenir à un état
            stable en cas de redémarrage intempestif pour une cause
            environnementale comme un EMP ou une alimentation électrique
            défaillante.
    
        \item[Évolutivité et maintenabilité:]\hfill\\
            le système doit pouvoir
            s'adapter facilement à de nombreuses autres sans demander de
            modification majeure.
            De plus il doit être aisé de modifier le système pour en
            améliorer les performances et les fonctionnalités.
            Il faut prendre en compte 3 types d'évolution:
            \begin{itemize}
                \item dimensionnelle,
                \item fonctionnelle,
                \item matérielle.
            \end{itemize}

        \item[Généricité:]\hfill\\
            il faut pouvoir décliner le système à moindre
            coût pour d'autres applications, donc prévoir en conséquence
            le paramétrage des modules, de l'interface...

        \item[Réutilisation:]\hfill\\
            réutiliser les composants existants et
            définir des composants susceptibles d'être réutilisés dans
            d'autres applications.

        \item[Traçabilité:] le système doit conserver pendant 2 ans au
            moins toutes les interventions et toutes les mesures.
    \end{description}

