\section{Spécification des APIs}

Pour une question de simplicité de communication, les fonctions des APIs sont 
spécifiées en langage C++.

\subsection{API Serveurs}
Les principales fonctions d'interfaçage du serveur SSH sont les suivantes:

\begin{itemize}
\item \textbf{InfoDataList getData(int deviceId, STATUS error);}
\begin{tabbing}
détails: \=\\
\>Demande les données de tous les capteurs de la station concernée. \\
entrée :  \\
\>int deviceId : identifiant du dispositif \\
sortie : \\
\>InfoDataList : liste chainé de la structure InfoData \\
\>STATUS error : retourne 0 si OK, sinon le code correspondant au type d'erreur rencontré \\
\end{tabbing}

\item \textbf{DevicePosition getPosition(int deviceId, STATUS error);}
\begin{tabbing}
détails: \=\\
\>Demande la position exacte de la station concernée. \\
entrée : \\
\>int deviceId : identifiant du dispositif \\
sortie : \\
\>DevicePosition : contient les informations de position du dispositif. \\
\>STATUS error : retourne 0 si OK, sinon le code correspondant au type d'erreur rencontré \\
\end{tabbing}


\item \textbf{DevicePower getPower(int deviceId, STATUS error);}
\begin{tabbing}
détails: \=\\
\>		Demande l’autonomie d'énergie de la station. \\
entrée : \\
\>		int deviceId : identifiant du dispositif \\
sortie : \\
\>		DevicePower : contient les informations de l’autonomie énergétique du dispositif. \\
\>		STATUS error : retourne 0 si OK, sinon le code correspondant au type d'erreur rencontré \\
\end{tabbing}

\item \textbf{STATUS configSensor(int deviceId, int sensorId, int min, int max);}
\begin{tabbing}
détails: \=\\
\>		Configure le seuil minimum et maximum critique du type de données concernées du capteur. \\
entrée : \\
\>		int deviceId : identifiant du dispositif \\
\>		int sensorId : identifiant du capteur à configurer \\
\>		int min : seuil minimum \\
\>		int max : seuil maximum \\
sortie : \\
\>		retourne le code d'erreur \\
\end{tabbing}

\item \textbf{STATUS reboot(int deviceId);}
\begin{tabbing}
détails: \=\\
\>		Rédemarre le système embarqué donné. \\
entrée : \\
\>		int deviceId : identifiant du dispositif \\ 
sortie : \\
\>		retourne le code d'erreur \\
\end{tabbing}

\item \textbf{STATUS check(int deviceId);}
\begin{tabbing}
détails: \=\\
\>		Vérifie le système \\
entrée : \\
\>		int deviceId : identifiant du dispositif \\
sortie : \\
\>		retourne le code d'erreur \\
\end{tabbing}

\item \textbf{STATUS showlog(int deviceId, FLAG options, char * fileName);}
\begin{tabbing}
détails: \=\\
\>		Enregistre le log dans le fichier indiqué (STDOUT pour la sortie standard) \\
entrée : \\
\>		int deviceId : identifiant du dispositif \\
\>		FLAGs : conditions de sélection et filtrage des alertes (InfoData) à afficher (ordre chronologique, inverse, type d'alerte, période...) \\
sortie : \\
\>		retourne le code d'erreur \\
\>		char * fileName : le chemin du fichier log à enregistrer \\
\end{tabbing}

\item \textbf{STATUS backup(char * path);}
\begin{tabbing}
détails: \=\\
\>		Enregistre un backUp de toutes les données dans le répertoire indiqué\\
entrée : \\
\>		char * path : répertoire dans lequelle sera fait le backup \\
sortie : \\
\>		retourne le code d'erreur \\
\end{tabbing}

\item \textbf{STATUS restore(char * path);}
\begin{tabbing}
détails: \=\\
\>		Restaure le système en utilisant le backup enregistré sur le répertoire indiqué \\
entrée : \\
\>		char * path : répertoire du backup \\
sortie : \\
\>		retourne le code d'erreur \\
\end{tabbing}

\end{itemize}



\subsection{API - Système embarqué}

Le système embarqué possède certaines fonctions similaires à l'API du serveur 
SSH, pour cela on utilisera les memes noms.

\begin{itemize}

\item \textbf{SensorData getData(int sensorId, STATUS error);}
\begin{tabbing}
détails: \=\\
\>		Demande les données d'un capteur \\
entrée : \\ 
\>		int sensorId : identifiant du capteur \\
sortie : \\
\>		SensorData : contient les informations transmises par le capteur \\
\>		STATUS error : retourne le code d'erreur \\
\end{tabbing}

\item \textbf{DevicePosition getPosition(STATUS error);}
\begin{tabbing}
détails: \=\\
\>		Demande la position exacte de la station concernée au module GPS. \\
sortie : \\
\>		DevicePosition : contient les informations de position du dispositif. \\
\>		STATUS error : retourne le code d'erreur \\
\end{tabbing}


\item \textbf{STATUS configSensor(int sensorId, int min, int max);}
\begin{tabbing}
détails: \=\\
\>		Configure le seuil minimum et maximum critique du type de données concernées du capteur. \\
entrée : \\
\>		int sensorId : identifiant du capteur à configurer \\
\>		int min : seuil minimum \\
\>		int max : seuil maximum \\
sortie : \\
\>		retourne le code d'erreur \\
\end{tabbing}

\item \textbf{STATUS reboot(int sensorId);}
\begin{tabbing}
détails: \=\\
\>		Rédemarre le capteur \\
entrée : \\
\>		int sensorId : identifiant du dispositif \\
sortie : \\
\>		retourne le code d'erreur \\
\end{tabbing}

\item \textbf{InfoData sendInfoData(int sensorId, STATUS error);}
\begin{tabbing}
détails: \=\\
\>		Envoie l'information sur un capteur \\
entrée : \\
\>		int sensorId : identifiant du capteur \\
sortie : \\
\>		InfoData - structure avec les informations relevé du capteur \\
\>		STATUS error : retourne le code d'erreur \\
\end{tabbing}

\end{itemize}

\subsection{API - PDA/Smartphone}

Aven un PDA/Smartphone, l'utilisateur doit être capable de consulter des informations 
et configurer le système. Pour cela, nous avons implementer un serveur web sur le système embarqué. \\
L'advantage de cette aproche est que nous n'avons pas besoin de créer/configurer un logiciel, il suffit que le PDA/Smartphone se connecte en bluetooth sur notre serveur web local avec n'importe quel browser.


\vfill
\pagebreak
