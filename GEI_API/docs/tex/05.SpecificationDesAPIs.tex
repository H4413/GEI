\section{Spécification des APIs}

Pour une question de simplicité de communication, les fonctions des APIs sont spécifié en langage C++.

\subsection{API Serveurs}
Les principales fonctions d'interfaçage du serveur SSH sont les suivantes :

\begin{enumerate}
\item \textbf{InfoDataList getData(int deviceId, STATUS error);}
détails:
		Demande les données de tous les capteurs de la station concernée.
entrée : 
		int deviceId : identifiant du dispositif
sortie :
		InfoDataList : liste chainé de la structure InfoData
		STATUS error : retourne 0 si OK, sinon le code correspondant au type d'erreur rencontré

\item \textbf{DevicePosition getPosition(int deviceId, STATUS error);}
détails:
		Demande la position exacte de la station concernée.
entrée : 
		int deviceId : identifiant du dispositif
sortie :
		DevicePosition : contient les informations de position du dispositif.
		STATUS error : retourne 0 si OK, sinon le code correspondant au type d'erreur rencontré

\item \textbf{DevicePower getPower(int deviceId, STATUS error);}
détails:
		Demande l’autonomie d'énergie de la station.
entrée : 
		int deviceId : identifiant du dispositif
sortie :
		DevicePower : contient les informations de l’autonomie énergétique du dispositif.
		STATUS error : retourne 0 si OK, sinon le code correspondant au type d'erreur rencontré

\item \textbf{STATUS configSensor(int deviceId, int sensorId, int min, int max);}
détails:
		Configure le seuil minimum et maximum critique du type de données concernées du capteur.
entrée :
		int deviceId : identifiant du dispositif
		int sensorId : identifiant du capteur à configurer
		int min : seuil minimum
		int max : seuil maximum
sortie :
		retourne le code d'erreur

\item \textbf{STATUS reboot(int deviceId);}
détails:
		Rédemarre le système embarqué donné.
entrée :
		int deviceId : identifiant du dispositif
sortie :
		retourne le code d'erreur

\item \textbf{STATUS check(int deviceId);}}
détails:
		Vérifie le système
entrée :
		int deviceId : identifiant du dispositif
sortie :
		retourne le code d'erreur

\item \textbf{STATUS showlog(int deviceId, FLAG options, char * fileName);}
détails:
		Enregistre le log dans le fichier indiqué (STDOUT pour la sortie standard)
entrée :
		int deviceId : identifiant du dispositif
		FLAGs : conditions de sélection et filtrage des alertes (InfoData) à afficher (ordre chronologique, inverse, type d'alerte, période...)
sortie :
		retourne le code d'erreur
		char * fileName : le chemin du fichier log à enregistrer

\item \textbf{STATUS backup(char * path);}
détails:
		Enregistre un backUp de toutes les données dans le répertoire indiqué
entrée :
		char * path : répertoire dans lequelle sera fait le backup
sortie :
		retourne le code d'erreur

\item \textbf{STATUS restore(char * path);}
détails:
		Restaure le système en utilisant le backup enregistré sur le répertoire indiqué
entrée :
		char * path : répertoire du backup
sortie :
		retourne le code d'erreur

\end{enumerate}


\subsection{API - Système embarqué}

Le système embarqué possède certaines fonctions similaires à l'API du serveur SSH, pour cela on utilisera les memes noms.

\begin{enumerate}

\item \textbf{SensorData getData(int sensorId, STATUS error);}
détails:
		Demande les données d'un capteur
entrée : 
		int sensorId : identifiant du capteur
sortie :
		SensorData : contient les informations transmises par le capteur 
		STATUS error : retourne le code d'erreur

\item \textbf{DevicePosition getPosition(STATUS error);}
détails:
		Demande la position exacte de la station concernée au module GPS.
sortie :
		DevicePosition : contient les informations de position du dispositif.
		STATUS error : retourne le code d'erreur


\item \textbf{STATUS configSensor(int sensorId, int min, int max);}
détails:
		Configure le seuil minimum et maximum critique du type de données concernées du capteur.
entrée :
		int sensorId : identifiant du capteur à configurer
		int min : seuil minimum
		int max : seuil maximum
sortie :
		retourne le code d'erreur

\item \textbf{STATUS reboot(int sensorId);}
détails:
		Rédemarre le capteur
entrée :
		int sensorId : identifiant du dispositif
sortie :
		retourne le code d'erreur

\item \textbf{InfoData sendInfoData(int sensorId, STATUS error);}
détails:
		Envoie l'information sur un capteur
entrée : 
		int sensorId : identifiant du capteur
sortie :
		InfoData - structure avec les informations relevé du capteur
		STATUS error : retourne le code d'erreur

\end{enumerate}

\subsection{API - PDA/Smartphone}

Le logiciel du PDA/Smartphone doit être capable de consulter des informations et configurer le système. Il utilise quelques fonctions identiques à l'API Serveur. Cettes fonctions sont les suivantes:


\begin{enumerate}
\item \textbf{InfoDataList getData(int deviceId, STATUS error);}
\item \textbf{DevicePosition getPosition(int deviceId, STATUS error);}
\item \textbf{DevicePower getPower(int deviceId, STATUS error);}
\item \textbf{STATUS configSensor(int deviceId, int sensorId, int min, int max);}
\item \textbf{STATUS reboot(int deviceId);}
\item \textbf{STATUS check(int deviceId);}
\end{enumerate}
