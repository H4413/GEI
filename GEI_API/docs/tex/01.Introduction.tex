\section{Introduction}

La société COPEVUE (Comité pour la Protection de l’EnVironnement de l’UE) s’occupe de la surveillance de sites isolés à distance dans des noumbreuses régions de l'UE auxquelles l’accès se réalise difficilement. La surveillance est basée sur les données recueillies par de différents capteurs installés sur site. Il est indispensable d’être capables de recevoir continûment les valeurs enregistrées par les capteurs afin de pouvoir intervenir au cas d’apparition de problèmes. \\
Le but de cette étude est d’offrir à COPEVUE une solution complète pour la surveillance de ses sites, ces lieux sont bien souvent isolés et disséminés loin des villes et des grands centres et doivent donc être autonomes en terme d’énergie, de déchets etc.
Dans un premier temps le dossier répondra le mieux possible à l’appel d’offres. Ce dossier se propose d’expliquer la façon dans laquelle les différents modules communiquent entre eux. \\
Nous avons aussi inclus une description de format des données, du flux des informations et des fonctions offertes par les différentes APIs

\vfill
\pagebreak
